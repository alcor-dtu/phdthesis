\documentclass[10pt,a4paper]{article}
\usepackage[utf8]{inputenc}
\usepackage{amsmath}
\usepackage{amsfonts}
\usepackage{amssymb}
\usepackage{xifthen}
\usepackage{hyperref}

\title{Camera space BSSRDF importance sampling}
\date{September 2017}
\author{Alessandro Dal Corso \\ Technical University of Denmark
}

\begin{document}
\maketitle
\newcommand{\vecfunc}[2] {\vec{#1}(\mathbf{#2})}
\newcommand{\func}[2] {{#1}(\mathbf{#2})}
\newcommand{\omegafunc}[2] {{#1}(\mathbf{#2}, \vec{\omega})}

\newcommand{\xvecfunc}[1] {\vecfunc{#1}{x}}
\newcommand{\xfunc}[1] {\func{#1}{x}}
\newcommand{\xomegafunc}[1] {\omegafunc{#1}{x}}
\newcommand{\nablavec} {{\nabla}}
\newcommand{\omegavec} {\vec{\omega}}
\newcommand{\sphere}[2] {\int_{4\pi}{#1}\ d{\ifthenelse{\isempty{#2}{}}{{\omega}}{#2}}}
\newcommand{\hemisphere}[2] {\int_{2\pi_+}{#1}\ d{\ifthenelse{\isempty{#2}{}}{{\omega}}{#2}}}
\newcommand{\inwardhemisphere}[2] {\int_{2\pi_-}{#1}\ d{\ifthenelse{\isempty{#2}{}}{{\omega}}{#2}}}

\newcommand{\absorption}{\sigma_a}
\newcommand{\transmission}{\sigma_{tr}}
\newcommand{\scattering}{\sigma_s}
\newcommand{\extinction}{\sigma_t}
\newcommand{\fluence}{G_0}
\newcommand{\flux}{\mathbf{G}_1}
\newcommand{\sourcezero}[1]{\func{q}{#1}}
\newcommand{\sourcezerox}{\xfunc{q}}
\newcommand{\sourceone}[1]{\vec{\mathbf{Q}}(\mathbf{#1}, \omegavec)}
\newcommand{\sourceonex}{\vec{\mathbf{Q}}(\mathbf{x}, \omegavec)}
\newcommand{\redsca}{{\sigma}'_s}
\newcommand{\redext}{{\sigma}'_t}
\newcommand{\redscaEddington}{\tilde{\sigma}_s}
\newcommand{\redextEddington}{\tilde{\sigma}_t}
\newcommand{\de}{\text{d}}
\newcommand{\cphi}{C_{\phi}}
\newcommand{\cE}{C_{\mathbf{E}}}
\newcommand{\nvec} {\vec{n}}
\newcommand{\radius} {r}

This note explain the importance sampling technique in camera space from Mertens et al.~\cite{Mertens2003}.

\section{Planar surfaces}
We start from the classical BSSRDF form of the rendering equation:
\begin{equation*}
L_o(\mathbf{x}_o, \omegavec_o) = \int_A \int_{\Omega(\mathbf{x}_i)} L_i(\mathbf{x}_i, \omegavec_i) S(\mathbf{x}_i, \omegavec_i, \mathbf{x}_o, \omegavec_o) (\omegavec_i \cdot \nvec_i) d\omega_i dA
\end{equation*}

Where $\Omega(\mathbf{x}_i)$ is an hemisphere placed in $\mathbf{x}_i$ with normal $\nvec_i$.
Our first step is to convert the integration into polar coordinates. We assume for now that the integration happens on a plane. In polar coordinates, we have $dA = r dr d\theta$. The integral becomes:

\begin{equation*}
L_o(\mathbf{x}_o, \omegavec_o) = \int_0^{2\pi} \int_0^{+\infty} \int_{\Omega(\mathbf{x}_i)} L_i(\mathbf{x}_i, \omegavec_i) S(\mathbf{x}_i, \omegavec_i, \mathbf{x}_o, \omegavec_o) (\omegavec_i \cdot \nvec_i) d\omega_i r dr d\theta
\end{equation*}
Where $r = \|\mathbf{x}_o-\mathbf{x}_i\|$. Also, $\mathbf{x}_i = \mathbf{x}_o + r \cos\theta\; \vec{t}_o + r \sin\theta\; \vec{b}_o$, where $\vec{t}_o$ and $\vec{b}_o$ form an orthonormal basis with $\vec{n}_o$.

We can now perform Monte Carlo integration, that gives the estimator:

\begin{equation*}
L_o^{N,M}(\mathbf{x}_o, \omegavec_o) = \frac{1}{NM}\sum_{p=1}^M \sum_{q=1}^N \frac{L_i(\mathbf{x}_{i,p}, \omegavec_{i,q}) S(\mathbf{x}_{i,p}, \omegavec_{i,q}, \mathbf{x}_o, \omegavec_o) (\omegavec_{i,q} \cdot \nvec_{i,p}) \radius }{\text{pdf}(\radius,\theta) \text{pdf}(\omega_{i,q})}
\end{equation*}
Now, we can importance sample our disc coordinates $(\radius,\theta)$ using the joint pdf:
$$
\text{pdf}(\radius,\theta) = \frac{1}{2 \pi} \sigma_{tr} e^{-\sigma_{tr} \radius} 
$$
And we can sample the incoming light direction using a cosine weighted hemisphere:
$$
\text{pdf}(\omega_{i,q}) = \frac{\omegavec_{i,q} \cdot \nvec_{i,p}}{\pi} 
$$
So the integral can be finally approximated as:
$$
L_o^{N,M}(\mathbf{x}_o, \omegavec_o) = \frac{2 \pi^2}{NM \sigma_{tr}}\sum_{p=1}^M \sum_{q=1}^N L_i(\mathbf{x}_{i,p}, \omegavec_{i,q}) S(\mathbf{x}_{i,p}, \omegavec_{i,q}, \mathbf{x}_o, \omegavec_o) \radius e^{\sigma_{tr} \radius}   
$$
Note that the  $\radius e^{\sigma_{tr} \radius} $ cancels out some terms inside the BSSRDF. 
\section{Non planar surfaces}
We now assume that the surface is no longer planar. So we want to make a change of variables so that our integral becomes tractable, i.e. we are able to perform it in the tangent plane. Let us call $dA_{tan}$ an element of area in the tangent plane. The change of variables simply becomes:

\begin{equation*}
L_o(\mathbf{x}_o, \omegavec_o) = \int_{A_{tan}} \int_{\Omega(\mathbf{x}_i)} L_i(\mathbf{x}_i, \omegavec_i) S(\mathbf{x}_i, \omegavec_i, \mathbf{x}_o, \omegavec_o) (\omegavec_i \cdot \nvec_i) d\omega_i \left|\frac{dA}{dA_{tan}} \right| dA_{tan}
\end{equation*}

Where $dA$ is the corresponding area element on the real surface, continuing a camera ray. We call the camera ray $\vec{d} = \frac{\mathbf{x}_i - \mathbf{e}}{\|\mathbf{x}_i - \mathbf{e}\|} $, where $\mathbf{e}$ is the camera position. 

We only need to estimate the Jacobian of the change of variables $\left|\frac{dA}{dA_{tan}} \right|$. We observe that, since we are using a pinhole camera, the solid angles subtended by the projected area elements must be equal. We calculate them and put them equal, that gives:
$$
\frac{dA (\nvec_i \cdot -\vec{d})}{d^2} = \frac{dA_{tan} (\nvec_o \cdot -\vec{d})}{d_{tan}^2} 
$$
Where $d = \|\mathbf{x}_i - \mathbf{e}\|$ and $d_{tan} = \|\mathbf{x}_{i,tan} - \mathbf{e}\|$ are the distances with the real point and the point in tangent space, respectively. The Jacobian then becomes:
$$
\left|\frac{dA}{dA_{tan}} \right| = \frac{(\nvec_o \cdot -\vec{d})}{(\nvec_i \cdot -\vec{d})} \frac{d^2}{d_{tan}^2}
$$
Note that for a plane, $\nvec_i = \nvec_o$ and $d = d_{tan}$, giving a unitary Jacobian as expected.

After the change of variables, we can now proceed as in the previous section, and obtaining the final estimator:
$$
L_o^{N,M}(\mathbf{x}_o, \omegavec_o) = \frac{2 \pi^2}{NM \sigma_{tr}}\sum_{p=1}^M \sum_{q=1}^N L_i(\mathbf{x}_{i,p}, \omegavec_{i,q}) S(\mathbf{x}_{i,p}, \omegavec_{i,q}, \mathbf{x}_o, \omegavec_o)\frac{(\nvec_o \cdot -\vec{d})}{(\nvec_i \cdot -\vec{d})} \frac{d^2}{d_{tan}^2}   \radius e^{\sigma_{tr} \radius} 
$$


\bibliographystyle{plain}

\bibliography{mertens_note}

\end{document}