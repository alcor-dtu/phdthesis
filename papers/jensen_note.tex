\documentclass[10pt,a4paper]{article}
\usepackage[utf8]{inputenc}
\usepackage{amsmath}
\usepackage{amsfonts}
\usepackage{amssymb}
\usepackage{xifthen}
\usepackage{hyperref}

\title{Derivation of standard and directional dipole quantities}
\date{March 2018}
\author{Alessandro Dal Corso \\ Technical University of Denmark \and Jeppe Revall Frisvad \\ Technical University of Denmark
\and Thomas Kim Kjeldsen \\ The Alexandra Institute
}

\begin{document}
\maketitle
\newcommand{\vecfunc}[2] {\mathbf{#1}(\mathbf{#2})}
\newcommand{\func}[2] {{#1}(\mathbf{#2})}
\newcommand{\omegafunc}[2] {{#1}(\mathbf{#2}, \vec{\omega})}


\newcommand{\xvecfunc}[1] {\vecfunc{#1}{x}}
\newcommand{\xfunc}[1] {\func{#1}{x}}
\newcommand{\xomegafunc}[1] {\omegafunc{#1}{x}}
\newcommand{\nablavec} {{\nabla}}
\newcommand{\omegavec} {\vec{\omega}}
\newcommand{\sphere}[2] {\int_{4\pi}{#1}\ d{\ifthenelse{\isempty{#2}{}}{{\omega}}{#2}}}
\newcommand{\hemisphere}[2] {\int_{2\pi_+}{#1}\ d{\ifthenelse{\isempty{#2}{}}{{\omega}}{#2}}}
\newcommand{\lowerhemisphere}[2] {\int_{2\pi_-}{#1}\ d{\ifthenelse{\isempty{#2}{}}{{\omega}}{#2}}}

\newcommand{\absorption}{\sigma_a}
\newcommand{\transmission}{\sigma_{tr}}
\newcommand{\scattering}{\sigma_s}
\newcommand{\extinction}{\sigma_t}
\newcommand{\fluence}{G_0}
\newcommand{\flux}{\mathbf{G}_1}
\newcommand{\sourcezero}[1]{\func{Q_0}{#1}}
\newcommand{\sourcezerox}{\xfunc{Q_0}}
\newcommand{\sourceone}[1]{\mathbf{Q}_1(\mathbf{#1})}
\newcommand{\sourceonex}{\mathbf{Q}_1(\mathbf{x})}
\newcommand{\redsca}{{\sigma}'_s}
\newcommand{\redext}{{\sigma}'_t}
\newcommand{\redscaEddington}{\tilde{\sigma}_s}
\newcommand{\redextEddington}{\tilde{\sigma}_t}
\newcommand{\de}{\text{d}}
\newcommand{\cphi}{C_{\phi}}
\newcommand{\cE}{C_{\mathbf{E}}}

\section{Integrating the radiative transfer equation}
We start from the radiative transfer equation~\cite{chandrasekhar50}:
$$
(\nablavec \cdot \omegavec) \xomegafunc{L}= -\extinction \xomegafunc{L} + \scattering \sphere {p(\omegavec', \omegavec) L(x,\omegavec')}{\omega'} + \xomegafunc{q} \, ,
$$
where $L$ is radiance at the position $\mathbf{x}$ in the direction $\omegavec$. The equation describes how the directional derivative of $L$ depends on the scattering properties of the surrounding medium, where $\extinction$ is the extinction coefficient, $\scattering$ is the scattering coefficient, and $p$ is the phase function. Finally, $q$ is the emitted radiance in the medium per unit length that we move along a ray.
If we then integrate over all directions $\omegavec$, we get
$$
\sphere{(\nablavec \cdot \omegavec) \xomegafunc{L}}{} =\sphere{ -\extinction \xomegafunc{L}}{} + \sphere{\scattering \sphere {p(\omegavec', \omegavec)L(x,\omegavec')}{\omega'}}{} + \sphere{\xomegafunc{q}}{} \, .
$$
Rearranging, we obtain
$$
\nablavec \cdot \left(\sphere{\omegavec \xomegafunc{L}}{}\right) = -\extinction \sphere{\xomegafunc{L}}{} + \scattering \sphere{\left(\sphere{p(\omegavec', \omegavec) }{}\right)\ L(x,\omegavec')}{\omega'} + \sourcezerox \, ,
$$
where we used the regularity of the operators to switch divergence and integral operations on the left-hand side, and to switch the integrals on the right-hand side. The integral of the phase function is 1, since it is normalized, so by further simplifying and applying the definitions of fluence $\phi$ and vector irradiance $\mathbf{E}$, we obtain
$$
\nablavec \cdot \xvecfunc{E} = -\extinction  \xfunc{\phi} + \scattering \sphere{L(x,\omegavec')}{\omega'} + \sourcezerox
$$
$$
\nablavec \cdot \xvecfunc{E} = -\extinction  \xfunc{\phi} + \scattering \xfunc{\phi} + \sourcezerox
$$
\begin{equation}
\nablavec \cdot \xvecfunc{E} = -\absorption  \xfunc{\phi} + \sourcezerox \, ,
\label{eq:rte}
\end{equation}
where we introduced the absorption coefficient $\absorption = \extinction - \scattering$. Our Equation~1 then corresponds to Equation~1 in the work of Jensen et al.~\cite{jensen01}.

\section{The diffusion approximation}

To get the diffusion approximation, we approximate radiance using a second order spherical harmonics expansion:
$$
\xomegafunc{L} \approx \sum_{n = 0}^{1} \sum_{m = -n}^{n} L_{n,m}(\mathbf{x}) Y_{n,m}(\omegavec) \, ,
$$
where $Y_{n,m}(\omegavec)$ are normalized real-valued spherical harmonics basis functions,
%\footnote{\url{https://www.cs.dartmouth.edu/~wjarosz/publications/dissertation/appendixB.pdf}},
and $L_{n,m}(\mathbf{x})$ is the projection of $L$ against the $n,m$ basis function:
$$
L_{n,m}(\mathbf{x}) = \sphere{\xomegafunc{L} Y_{n,m}(\omegavec)}{} \, .
$$
For $n = 0$, the integral is trivial.  The first term of the sum becomes:
$$
L_{0,0}(\mathbf{x}) Y_{0,0}(\omegavec) = \sphere{\sqrt{\frac{1}{4\pi}} \xomegafunc{L}}{} \, \sqrt{\frac{1}{4\pi}} = \frac{1}{4\pi} \xfunc{\phi} \, .
$$
As for the other basis functions, we use the Cartesian form. We have
$$
L_{1,-1}(\mathbf{x}) Y_{1,-1}(\omegavec) = \omega_x \sphere{\sqrt{\frac{3}{4\pi}} \omega_x \xomegafunc{L}}{} \, \sqrt{\frac{3}{4\pi}} = \frac{3}{4\pi} \omega_x {E}_x(\mathbf{x}) \, ,
$$
where the $x$ subscript indicates the first component. Similarly, we obtain
$$
L_{1,0}(\mathbf{x}) Y_{1,0}(\omegavec) = \frac{3}{4\pi} \omega_z {E}_z(\mathbf{x})
$$
$$
L_{1,1}(\mathbf{x}) Y_{1,1}(\omegavec) = \frac{3}{4\pi} \omega_y {E}_y(\mathbf{x}) \, .
$$
By applying the approximation, we finally obtain:
$$
\xomegafunc{L} \approx \frac{1}{4\pi} \xfunc{\phi} + \frac{3}{4\pi} \omega_x {E}_x(\mathbf{x}) + \frac{3}{4\pi} \omega_y {E}_y(\mathbf{x})+ \frac{3}{4\pi} \omega_z {E}_z(\mathbf{x}) = \frac{\xfunc{\phi}}{4\pi} + \frac{3}{4\pi} \omegavec \cdot \xvecfunc{E} \, .
$$

\section{The diffusion equation}

To find an equation for the relation between the fluence and the optical properties of the medium, we substitute the diffusion approximation into the radiative transfer equation:
\begin{multline*}
(\nablavec \cdot \omegavec) \left(\frac{\xfunc{\phi}}{4\pi} + \frac{3}{4\pi} \omegavec \cdot \xvecfunc{E}\right)
= -\extinction\left(\frac{\xfunc{\phi}}{4\pi} + \frac{3}{4\pi} \omegavec \cdot \xvecfunc{E}\right)
 \\+ \scattering \sphere {p(\omegavec', \omegavec) \left(\frac{\xfunc{\phi}}{4\pi} + \frac{3}{4\pi} \omegavec' \cdot\xvecfunc{E}\right)}{\omega'}
 + \xomegafunc{q} \, .
\end{multline*}
For simplification, we need the following three identities:
$$
\sphere{\omegavec}{} = 0
$$
\begin{equation}
\sphere{\omegavec (\omegavec \cdot \mathbf{A}) }{} = \frac{4 \pi}{3} \mathbf{A}
\label{eq:id1}
\end{equation}
$$
\sphere{\omegavec [ \omegavec \cdot \nablavec(\omegavec \cdot \mathbf{A})]}{} = 0
$$
We first multiply into parentheses in the equation above:
\begin{multline*}
\frac{1}{4\pi} \omegavec \cdot \nablavec \xfunc{\phi} + \frac{3}{4\pi} \omegavec \cdot \nablavec (\omegavec \cdot \xvecfunc{E})
= -\extinction\frac{\xfunc{\phi}}{4\pi}  -\extinction\frac{3}{4\pi} \omegavec \cdot \xvecfunc{E} + \scattering \frac{\xfunc{\phi}}{4\pi} \sphere {p(\omegavec', \omegavec)}{\omega'}
 \\+ \frac{3}{4\pi} \scattering  \sphere {p(\omegavec', \omegavec)  \omegavec' \cdot \xvecfunc{E}}{\omega'}
 + \xomegafunc{q} \, .
\end{multline*}
Now, we multiply each term by $\omegavec$ and integrate over the sphere. Taking all the terms separately, we have
$$
\sphere{\frac{1}{4\pi} \omegavec \cdot \nablavec \xfunc{\phi}\omegavec}{} = \frac{1}{4\pi} \frac{4\pi}{3} \nablavec \xfunc{\phi} = \frac{\nablavec \xfunc{\phi}}{3}
$$
$$
\sphere{\frac{3}{4\pi} \omegavec \cdot \nablavec (\omegavec \cdot \xvecfunc{E}) \omegavec}{} = \frac{3}{4\pi} \sphere{\omegavec [\omegavec \cdot \nablavec (\omegavec \cdot \xvecfunc{E})] }{} = 0
$$
$$
\sphere{-\extinction\frac{\xfunc{\phi}}{4\pi} \omegavec}{} = -\extinction\frac{\xfunc{\phi}}{4\pi} \sphere{\omegavec}{} =  0
$$
$$
\sphere{-\extinction\frac{3}{4\pi} \omegavec \cdot \xvecfunc{E} \omegavec}{} = -\extinction \frac{3}{4\pi} \sphere{ \omegavec (\omegavec \cdot \xvecfunc{E})}{} = -\extinction\frac{3}{4\pi} \frac{4\pi}{3} \xvecfunc{E} = -\extinction \xvecfunc{E}
$$
$$
\sphere{\scattering \frac{\xfunc{\phi}}{4\pi} \sphere {p(\omegavec', \omegavec)}{\omega'} \omegavec}{} = \scattering \frac{\xfunc{\phi}}{4\pi} \sphere{\omegavec}{} = 0
$$
$$
\sphere{ \frac{3}{4\pi} \scattering  \sphere {p(\omegavec', \omegavec)  \omegavec' \cdot \xvecfunc{E}}{\omega'} \omegavec}{} \stackrel{(*)}{=} g \scattering  \xvecfunc{E}
$$
$$
\sphere{ \xomegafunc{q} \omegavec}{} =  \sourceonex \, ,
$$
Passage $(*)$ is a bit more delicate as it requires the assumption that the phase function is rotationally symmetric, which means that it depends only on the cosine between the two direction vector arguments ($p(\omegavec', \omegavec) = p(\omegavec' \cdot \omegavec)$). For the interested reader, we further discuss this result at the end of this section (Section~\ref{sec:furtherderiv}). Putting everything together:
$$
\frac{\nablavec \xfunc{\phi}}{3} + 0 = 0 -\extinction \xvecfunc{E} + 0 + g \scattering  \xvecfunc{E} + \sourceonex
$$
\begin{equation}
\nablavec \xfunc{\phi} = -3\extinction' \xvecfunc{E} + 3\sourceonex \, ,
\label{eq:diff}
\end{equation}
where we used $\extinction' = \scattering' + \absorption = \scattering (1-g) + \absorption = \extinction -g \scattering$.
To obtain the diffusion equation, we need to combine Equations~\ref{eq:diff} and \ref{eq:rte}. We first rearrange Equation~\ref{eq:diff}:
$$
\xvecfunc{E} = 3 D \sourceonex  - D \nablavec \xfunc{\phi} \, ,
$$
where $D = \frac{1}{3 \extinction'}$. Inserting into Equation~\ref{eq:rte} (and assuming a homogeneous medium), we have
$$
\nablavec \cdot (3 D \sourceonex  - D \nablavec \xfunc{\phi}) = -\absorption  \xfunc{\phi} + \sourcezerox
$$
$$
3 D \nablavec \cdot \sourceonex - D \nabla^2 \xfunc{\phi} =  -\absorption  \xfunc{\phi} + \sourcezerox
$$
\begin{equation} \label{eq:diffusion}
D \nabla^2 \xfunc{\phi} = \absorption \xfunc{\phi} - \sourcezerox + 3 D \nablavec \cdot \sourceonex \, ,
\end{equation}
which is the diffusion equation as it appears in the work of Jensen et al.~\cite{jensen01}.

\subsection{Rotationally symmetric phase function}
\label{sec:furtherderiv}
Assuming that $p$ is rotationally symmetric so that $p(\omegavec', \omegavec) = p(\omegavec' \cdot \omegavec)$, we can expand it in Legendre polynomials:
$$
p(\omegavec' \cdot \omegavec) = \sum_{n = 0}^{\infty} \frac{2n+1}{4\pi} p_{n} P_n(\omegavec' \cdot \omegavec) \,,
$$
where $p_n$ are the expansion coefficients and $P_n$ are the Legendre polynomials. Since
$$
P_0(\omegavec' \cdot \omegavec) = 1, \quad P_1(\omegavec' \cdot \omegavec) = \omegavec' \cdot \omegavec  \, ,
$$
we have
$$
p_0 = \sphere{p(\omegavec' \cdot \omegavec) P_0(\omegavec' \cdot \omegavec)}{} = 1 \, ,
$$
$$
p_1 = \sphere{p(\omegavec' \cdot \omegavec) P_1(\omegavec' \cdot \omegavec)}{} = \sphere{p(\omegavec' \cdot \omegavec) (\omegavec' \cdot \omegavec)}{} = g \, .
$$
In addition, the orthogonality relations for Legendre polynomials are
$$
\sphere{P_n(\omegavec' \cdot \omegavec)P_m(\omegavec' \cdot \omegavec)}{\omega'} = \left\{\begin{array}{c@{\quad,\quad}l}\displaystyle\frac{4\pi}{2n+1} & \text{for $n=m$} \\[2ex] 0 & \text{otherwise} \, . \end{array} \right.
$$
Noting that in spherical harmonics with $\omegavec$ as the local $z$-axis:
\begin{eqnarray*}
\omega_x' & = & -\sqrt{\frac{4\pi}{3}}Y_{1,1}(\omegavec') \\
\omega_y' & = & -\sqrt{\frac{4\pi}{3}}Y_{1,-1}(\omegavec') \\
\omega_z' & = & \sqrt{\frac{4\pi}{3}}Y_{1,0}(\omegavec') = P_1(\omegavec' \cdot \omegavec) \, ,
\end{eqnarray*}
and that integration over $\omega_x'$ and $\omega_y'$ are zero, the expansion of the phase function combined with the orthogonality of the Legendre polynomials leave us with
\begin{equation*}
\begin{split}
&\sphere{ \frac{3}{4\pi} \scattering  \sphere {p(\omegavec', \omegavec)  \omegavec' \cdot \xvecfunc{E}}{\omega'} \omegavec}{} \\
&= \sphere{ \frac{3}{4\pi} \scattering \frac{3}{4\pi}g \left(\frac{4\pi}{3}\omegavec\cdot\xvecfunc{E}\right) \omegavec}{} = g \scattering  \xvecfunc{E} \, ,
\end{split}
\end{equation*}
which is the expected result.
\section{Boundary condition}

In the case of a scattering medium in a half-space, we impose the classic boundary condition that the net inward flux on each surface point $\mathbf{x}_s$ with (inward) normal $\vec{n}_s$ is zero:
$$
\hemisphere{L(\mathbf{x}_s, \omegavec) (\omegavec \cdot \vec{n}_s)}{} = 0 \, .
$$
We use the diffusion approximation:
$$
\hemisphere{\left(\frac{\func{\phi}{\mathbf{x}_s}}{4\pi} + \frac{3}{4\pi} \omegavec \cdot \vecfunc{E}{\mathbf{x}_s}\right) (\omegavec \cdot \vec{n}_s)}{} = 0
$$
$$
\xfunc{\phi} \hemisphere{ (\omegavec \cdot \vec{n}_s)}{} + 3  \hemisphere{(\omegavec \cdot \vecfunc{E}{\mathbf{x}_s}) (\vec{n}_s \cdot \omegavec) }{} = 0 \, .
$$
Given the standard spherical coordinates convention, $n_s = (0,0,1)$ and $\omegavec = (\cos\phi\sin\theta,\sin\phi\sin\theta,\cos\theta)$. We then obtain
$$
\hemisphere{ (\omegavec \cdot \vec{n}_s)}{} = \int_{0}^{2\pi} \int_{0}^{\frac{\pi}{2}} \cos\theta \sin\theta d\theta d\phi = \pi \, ,
$$
and
$$
\hemisphere{(\omegavec \cdot \xvecfunc{E}) (\vec{n}_s \cdot \omegavec) }{} = \int_{0}^{2\pi} \int_{0}^{\frac{\pi}{2}}(\cos\phi\sin\theta E_x+\sin\phi\sin\theta E_y+\cos\theta E_z) \cos\theta \sin\theta d\theta d\phi
$$
$$
= \frac{2\pi}{3} E_z = \frac{2\pi}{3} \vec{n}_s \cdot \vecfunc{E}{\mathbf{x}_s} \, .
$$
Using the last two results and simplifying, we get:
$$
\xfunc{\phi} \pi + 3 \left(\frac{2\pi}{3} \vec{n}_s \cdot \vecfunc{E}{\mathbf{x}_s}\right)= 0
$$
$$
\xfunc{\phi} + 2 \vec{n}_s \cdot \vecfunc{E}{\mathbf{x}_s} = 0 \, .
$$
From Equation \ref{eq:diff}, assuming no emission in $\mathbf{x}_s$, we have $\sourceone{\mathbf{x}_s} = \mathbf{0}$, so
$$
\vecfunc{E}{\mathbf{x}_s} = - D\nablavec \func{\phi}{\mathbf{x}_s} \, .
$$
Inserting this result and simplifying, we get the final boundary condition:
$$
\func{\phi}{\mathbf{x}_s} - 2 D (\vec{n}_s \cdot \nablavec) \func{\phi}{\mathbf{x}_s} = 0 \, .
$$

\section{Different media assumption}

To include nonzero inward flux at boundaries, we need to change the above equations. The boundary condition then becomes:
\begin{equation}
I_+ = \hemisphere{L(\mathbf{x}_s, \omegavec) (\omegavec \cdot \vec{n}_s)}{} = \lowerhemisphere{R(\eta, \omegavec) L(\mathbf{x}_s, \omegavec) (-\omegavec \cdot \vec{n}_s)}{} = I_- \, .
\label{eq:diffb}
\end{equation}
Keeping the above conventions, we define the Fresnel reflectance $R$ by:
$$
R(\eta, \omegavec) =
\left\{\begin{array}{c@{\quad}l}
1 & \text{for $\frac{\pi}{2} \leq \theta \leq \pi - \theta_c$}\\
F_r(\eta, \omegavec\cdot\vec{n}_s) & \text{for $\pi - \theta_c \leq \theta \leq \pi$} \, ,
\end{array} \right.
$$
where $\theta_c$ is the critical angle, and
$$
F_r(\eta, \mu) = \frac{1}{2}\left[\left(\frac{\mu - \eta\mu_0}{\mu + \eta\mu_0}\right)^{\!2} + \left(\frac{\eta\mu - \mu_0}{\eta\mu + \mu_0}\right)^{\!2}\right] \quad \text{with} \quad \mu_0^2 = 1 - \eta^2(1 - \mu^2)
$$
and $\cos^2\theta_c = \max(1 - \eta^{-2}, 0)$.
%We use $\pi - \theta_c$ in the definition of $R$, as the critical angle is conventionally defined from the normal pointing \textit{inside} the surface.
In principle, if we allow complex numbers, we would have $R(\eta, \omegavec) = F_r(\eta, \omegavec\cdot\vec{n}_s)$.

The left side of Equation~\ref{eq:diffb} is:
$$
I_+ = \frac{1}{4}(\xfunc{\phi} - 2 D (\vec{n}_s \cdot \nablavec) \xfunc{\phi}) \, .
$$
The other side is more tricky, since it requires splitting the integration along the different angles. We proceed as before, introducing the diffusion approximation:
$$
I_- = \frac{\xfunc{\phi}}{4\pi} \lowerhemisphere{ R(\eta, \omegavec) (-\omegavec \cdot \vec{n}_s)}{} + \frac{3}{4\pi} \lowerhemisphere{R(\eta, \omegavec)(\omegavec \cdot \xvecfunc{E}) (-\omegavec \cdot \vec{n}_s) }{} \, .
$$
The cosine-weighted integration of the Fresnel reflectance is sometimes referred to as diffuse Fresnel reflectance $F_{dr}$. If we, outside the region of total internal reflection, approximate the $R$ function by Fresnel reflectance for normal incidence $R_0 = F_r(\eta, 1)$, we can find an approximate analytical solution. The first part is then:
\begin{multline*}
\lowerhemisphere{ R(\eta, \omegavec) (-\omegavec \cdot \vec{n}_s)}{} = \int_{0}^{2\pi} \int_{\frac{\pi}{2}}^{\pi - \theta_c} (-\cos\theta) \sin\theta d\theta d\phi + \int_{0}^{2\pi} \int_{\pi - \theta_c}^{\pi} R_0 (-\cos\theta) \sin\theta d\theta d\phi \\ = \pi((1 - R_0)\cos^2\theta_c + R_0) \, .
\end{multline*}
The second part (only on the $z$-coordinate, since the other coordinates are zero):
\begin{multline*}
\lowerhemisphere{R(\eta, \omegavec)(\omegavec \cdot \xvecfunc{E}) (-\omegavec \cdot \vec{n}_s) }{}\\ {}=E_z \int_{0}^{2\pi} \int_{\frac{\pi}{2}}^{\pi - \theta_c} (-\cos^2\theta) \sin\theta d\theta d\phi + E_z \int_{0}^{2\pi}  \int_{\pi - \theta_c}^{\pi} R_0 (-\cos^2\theta) \sin\theta d\theta d\phi \\
{}=\frac{2\pi}{3} E_z (R_0 (\cos^3\theta_c - 1) - \cos^3\theta_c) \, .
\end{multline*}
Performing all simplifications, we finally get:
$$
I_- =  \frac{1}{4}[((1 - R_0)\cos^2\theta_c + R_0) \phi(x) - 2 D (R_0 (\cos^3\theta_c - 1) - \cos^3\theta_c) (\vec{n}_s \cdot \nablavec) \xfunc{\phi}] \, .
$$
We can now impose $I_+ = I_-$:
$$
 \xfunc{\phi} - 2 D (\vec{n}_s \cdot \nablavec) \xfunc{\phi} = ((1 - R_0)\cos^2\theta_c + R_0) \phi(x) - 2 D (R_0 (\cos^3\theta_c - 1) - \cos^3\theta_c) (\vec{n}_s \cdot \nablavec) \xfunc{\phi} \, ,
$$
which we can simplify as follows:
$$
  \xfunc{\phi} - 2 \frac{1 + R_0 + (1 - R_0)\cos^3\theta_c}{1 - R_0 - (1 - R_0)\cos^2\theta_c} D (\vec{n}_s \cdot \nablavec) \xfunc{\phi} = 0
$$
$$
  \xfunc{\phi} - 2 \frac{\frac{1 + R_0}{1 - R_0} + \cos^3\theta_c}{1 - \cos^2\theta_c} D (\vec{n}_s \cdot \nablavec) \xfunc{\phi} = 0
$$
$$
  \xfunc{\phi} - 2 A D (\vec{n}_s \cdot \nablavec) \xfunc{\phi} = 0 \, .
$$
So, to handle reflective boundaries, we need to add a correction factor $A$ in our boundary condition. With our current approximation, we have
$$
A = \frac{\frac{1 + R_0}{1 - R_0} + \cos^3\theta_c}{1 - \cos^2\theta_c} = \frac{\frac{\eta^2 + 1}{2\eta} + [\max(1 - \eta^{-2}, 0)]^{\frac{3}{2}}}{1 - \max(1 - \eta^{-2}, 0)} \, .
$$

\subsection{Approximating the corrective factor}

Assuming separability, we can rewrite the $I_-$ term in Equation~\ref{eq:diffb} as:
$$
 \lowerhemisphere{R(\eta, \omegavec) L(x_s, \omegavec) (-\omegavec \cdot \vec{n}_s)}{} \approx \lowerhemisphere{R(\eta, \omegavec)  (-\omegavec \cdot \vec{n}_s)}{} \lowerhemisphere{L(x_s, \omegavec) (-\omegavec \cdot \vec{n}_s)}{}
$$
$$
= F_{dr}(\eta)\  \frac{1}{4}(\xfunc{\phi} + 2 D (\vec{n}_s \cdot \nablavec) \xfunc{\phi})
$$
An approximate fit of the $F_{dr}(\eta)$ integral is~\cite{egan73}
$$
F_{dr}(\eta) = \left\{\begin{array}{ccl} \displaystyle -0.4399 + \frac{0.7099}{\eta} - \frac{0.3319}{\eta^2} + \frac{0.0636}{\eta^3} & , & \eta < 1 \\[2ex]
\displaystyle -\frac{1.4399}{\eta^2} +\frac{0.7099}{\eta} + 0.6681 + 0.0636 \eta & , & \eta > 1 \, .\end{array}\right.
$$
So we can express the boundary condition as
$$
\xfunc{\phi} - 2 \pi D (\vec{n}_s \cdot \nablavec) \xfunc{\phi} = F_{dr}(\eta)\ (\xfunc{\phi} + 2 D (\vec{n}_s \cdot \nablavec) \xfunc{\phi})
$$
$$
\xfunc{\phi} (1 - F_{dr}) - 2 D (1 + F_{dr}) (\vec{n}_s \cdot \nablavec) \xfunc{\phi} = 0
$$
$$
\xfunc{\phi} - 2 D \frac{1 + F_{dr}}{1 - F_{dr}} (\vec{n}_s \cdot \nablavec) \xfunc{\phi} = 0
$$
$$
\xfunc{\phi} - 2 A D (\vec{n}_s \cdot \nablavec) \xfunc{\phi} = 0
$$
with $A = \frac{1 + F_{dr}}{1 - F_{dr}}$, which returns values fairly close to the $A$ found in the previous section. The $A$ in this section is the one employed by Jensen et al.~\cite{jensen01}. With respect to $F_{dr}$, they only provide the more common case of $\eta > 1$.

\section{Solutions for an infinite medium}
From the diffusion equation (\ref{eq:diffusion}), we have
$$
(D \nabla^2 - \absorption) \xfunc{\phi} =  -\sourcezerox + 3 D \nablavec \cdot \sourceonex
$$
$$
(\nabla^2 - \transmission^2) \xfunc{\phi} =  -  \frac{\sourcezerox}{D} + 3 \nablavec \cdot \sourceonex \, ,
$$
which is a particular case of the screened Poisson equation. This has a generic solution based on the method of Green's functions:
$$
\xfunc{\phi} = \frac{1}{4 \pi} \iiint_{\mathbb{R}^3} \frac{e^{-\transmission \|\mathbf{x} - \mathbf{r}' \|}}{\|\mathbf{x} - \mathbf{r}' \|} \left(  \frac{\sourcezero{r'}}{D} - 3 \nablavec \cdot \sourceone{r'}\right) d^3 \mathbf{r}' \, .
\label{eq:poisson}
$$
\subsection{Point source solutions}
If we use a point source placed at the origin, we have
$$
\sourcezerox = \Phi_i \delta(\mathbf{x})
$$
$$
\sourceonex = 0 \, .
$$
Inserting in the solution based on Green's function:
$$
\xfunc{\phi} = \frac{1}{4 \pi} \iiint_{\mathbb{R}^3} \frac{e^{-\transmission \|\mathbf{x} - \mathbf{r}' \|}}{\|\mathbf{x} - \mathbf{r}' \|} \left(\frac{\Phi_i \delta(\mathbf{r}')}{D}\right) d^3 \mathbf{r}'
$$
and applying the delta function, the result is
$$
\xfunc{\phi} = \frac{\Phi_i}{4 \pi D} \frac{e^{-\transmission r}}{r} \, ,
$$
where $r = \|\mathbf{x}\|$ is the distance to the point of interest. A similar result is obtained if we consider a ray source at the origin with direction along the inward surface normal ($z$-axis). Suppose the medium exhibits isotropic scattering, then the source of first scattering events is~\cite{patterson89}
$$
\sourcezerox = \Phi_i \redsca \delta(x) \delta(y) \Theta(z) e^{-\redext z}
$$
$$
\sourceonex = 0 \, ,
$$
where $\Theta(z)$ is the Heaviside step function, which is 1 for $z \geq 0$ and  0 otherwise. Inserting in the solution based on Green's function with $\mathbf{r}' = (x', y', z')$:
$$
\xfunc{\phi} = \frac{1}{4 \pi} \iiint_{\mathbb{R}^3} \frac{e^{-\transmission \|\mathbf{x} - \mathbf{r}' \|}}{\|\mathbf{x} - \mathbf{r}' \|} \left(   \frac{\Phi_i \redsca \delta(x') \delta(y') \Theta(z') e^{-\redext z'}}{D} \right) d^3 \mathbf{r}'
$$
and applying the deltas:
$$
\xfunc{\phi} = \frac{\Phi_i}{4 \pi D} \int_{0}^{+\infty} \frac{e^{-\transmission \|\mathbf{x} - z' \vec{n}_s \|}}{\|\mathbf{x} - z' \vec{n}_s  \|} \left(  \redsca e^{-\redext z'} \right) dz' \, .
$$
Considering positions $\mathbf{x}$ in the $xy$-plane far from the origin and the exponential attenuation with increasing $z'$, we use the assumption $\|\mathbf{x} - z' \vec{n}_s \| \approx \|\mathbf{x}\| = r$ and get
$$
\xfunc{\phi} = \frac{\Phi_i}{4 \pi D} \frac{e^{-\transmission r}}{r} \redsca \int_{0}^{+\infty} e^{-\redext z'} dz' = \frac{\Phi_i}{4 \pi D} \frac{e^{-\transmission r}}{r} \frac{\redsca}{\redext} = \alpha' \frac{\Phi_i}{4 \pi D} \frac{e^{-\transmission r}}{r} \, ,
$$
where $\alpha' = \frac{\redsca}{\redext}$ is the reduced scattering albedo. Thus, we get the monopole solution for a ray source of normal incidence:
\begin{equation} \label{eq:monopoint}
\xfunc{\phi} = \frac{\Phi}{4 \pi D} \frac{e^{-\transmission r}}{r} \qquad \text{with $\Phi = \alpha' \Phi_i$.}
\end{equation}

\subsection{Ray source solution}

In case of a ray source that is not along the normal direction and not necessarily in an isotropic medium, we can use the following equations for the source terms~\cite{menon05}:
$$
\sourcezerox = \Phi_i \redscaEddington \delta(x) \delta(y) \Theta(z) e^{-\redextEddington z}
$$
$$
\sourceonex = \tilde{g} \sourcezerox \vec{n}_s \, ,
$$
where we have used the delta-Eddington scattering properties~\cite{joseph76}:
$$
\tilde{\sigma}_s = \sigma_s(1 - g^2) \quad , \quad \tilde{\sigma}_t = \tilde{\sigma}_s + \sigma_a \quad , \quad \tilde{g} = g/(g + 1) \, .
$$
Inserting in the diffusion equation (\ref{eq:diffusion}), we get two integrals when using the Green's function solution. Splitting the solution accordingly: $\xfunc{\phi} = \xfunc{\phi_1} + \xfunc{\phi_2}$, we have
$$
\xfunc{\phi_1} = \frac{1}{4 \pi} \iiint_{\mathbb{R}^3} \frac{e^{-\transmission \|\mathbf{x} - \mathbf{r}' \|}}{\|\mathbf{x} - \mathbf{r}' \|} \left(   \frac{\Phi_i \redscaEddington \delta(x') \delta(y') \Theta(z') e^{-\redextEddington z'}}{D} \right) d^3 \mathbf{r}'
$$
$$
= \frac{\Phi_i \redscaEddington}{4 \pi D} \int_{0}^{+\infty} \frac{e^{-\transmission \|\mathbf{x} - z' \vec{n}_s\|}}{\|\mathbf{x} - z' \vec{n}_s\|}  e^{-\redextEddington z'} dz'
$$
$$
= \frac{3\redscaEddington \Phi_i}{4 \pi} \redextEddington \int_{0}^{+\infty} \frac{e^{-\transmission \|\mathbf{x} - z' \vec{n}_s\|}}{\|\mathbf{x} - z' \vec{n}_s\|}  e^{-\redextEddington z'} dz'
$$
and
$$
\xfunc{\phi_2} = -\frac{3}{4 \pi} \iiint_{\mathbb{R}^3}  \frac{e^{-\transmission \|\mathbf{x} - \mathbf{r}' \|}}{\|\mathbf{x} - \mathbf{r}' \|} \nabla \cdot \left(  \Phi_i \redscaEddington \tilde{g} \delta(x') \delta(y') \Theta(z') e^{-\redextEddington z'} \right)\! \vec{n}_s \, d^3 \mathbf{r}' \, .
$$

We now apply the divergence operator:
$$
\xfunc{\phi_2} = -\frac{3 \Phi_i \redscaEddington \tilde{g}}{4 \pi} \iiint_{\mathbb{R}^3}  \frac{e^{-\transmission \|\mathbf{x} - \mathbf{r}' \|}}{\|\mathbf{x} - \mathbf{r}' \|} \left[ \delta(x) \delta(y)
\frac{\partial}{\partial z'}  (\Theta(z) e^{-\redextEddington z'}) \right]  d^3 \mathbf{r}'.
$$
Note that we have only the $z$-term given that we multiply by $\vec{n}_s = (0,0,1)$. Using that $\frac{\partial \Theta(z')}{\partial z'} = \delta(z') $, we have
$$
\xfunc{\phi_2} = -\frac{3 \Phi_i \redscaEddington \tilde{g}}{4 \pi} \iiint_{\mathbb{R}^3}  \frac{e^{-\transmission \|\mathbf{x} - \mathbf{r}' \|}}{\|\mathbf{x} - \mathbf{r}' \|}  \delta(x') \delta(y') \left[
\delta(z') e^{-\redextEddington z'} - \redextEddington \Theta(z') e^{-\redextEddington z'} \right]  d^3 \mathbf{r}'.
$$
Applying the deltas, we get
$$
\xfunc{\phi_2} = -\frac{3 \Phi_i \redscaEddington \tilde{g}}{4 \pi} \frac{e^{-\transmission r}}{r} +\frac{3 \Phi_i \redscaEddington \tilde{g} \redextEddington}{4 \pi} \iiint_{\mathbb{R}^3}  \frac{e^{-\transmission \|\mathbf{x} - \mathbf{r}' \|}}{\|\mathbf{x} - \mathbf{r}'\|} \delta(x') \delta(y') \Theta(z') e^{-\redextEddington z'}  d^3 \mathbf{r}'
$$
$$
= -\frac{3 \Phi_i \redscaEddington \tilde{g}}{4 \pi} \frac{e^{-\transmission r}}{r} +\frac{3 \Phi_i \redscaEddington \tilde{g} \redextEddington}{4 \pi} \int_{0}^{+\infty} \frac{e^{-\transmission \|\mathbf{x} - z' \vec{n}_s\|}}{\|\mathbf{x} - z' \vec{n}_s\|}  e^{-\redextEddington z'} dz' \, .
$$
Putting it together:
$$
\xfunc{\phi} = -\frac{3 \Phi_i \redscaEddington \tilde{g}}{4 \pi} \frac{e^{-\transmission r}}{r} + \frac{3\redscaEddington \Phi_i}{4 \pi} (\redextEddington + \redscaEddington \tilde{g} + \absorption  \tilde{g} ) \int_{0}^{+\infty} \frac{e^{-\transmission \|\mathbf{x} - z' \vec{n}_s\|}}{\|\mathbf{x} - z' \vec{n}_s\|}  e^{-\redextEddington z'}  dz'
$$
\begin{equation}
\xfunc{\phi} = \frac{3 \Phi_i \redscaEddington}{4 \pi} \left(-\tilde{g} \frac{e^{-\transmission r}}{r} + (\redscaEddington + \absorption  ( 1 + \tilde{g})) \int_{0}^{+\infty} \frac{e^{-\transmission \|\mathbf{x} - z' \vec{n}_s\|}}{\|\mathbf{x} - z' \vec{n}_s\|}  e^{-\redextEddington z'} dz' \right) .
\label{eq:phi}
\end{equation}

We can interpret the second term in Eq.~\ref{eq:phi} as the fluence from
an exponentially decaying line source along the $z$-axis.
Due to this exponentially decaying factor, the integrand
will only have a significant weight for small $z$.
Hence, in the asymptotic limit, $r \gg 1/\redscaEddington$, we can approximate
the distances in the integrand
\begin{eqnarray*}
    \|\mathbf{x} - z' \vec{n}_s\| &=& \sqrt{r^2 + z'^2 - 2 z' r \cos
    \theta} = r \left(1 - 2 \frac{z'}{r} \cos \theta + \frac{z'^2}{r^2}
    \right)^\frac{1}{2} \\
    &\approx& r \left( 1 -  \frac{z'}{r} \cos \theta \right) = r - z' \cos
    \theta \, ,
\end{eqnarray*}
and
\begin{eqnarray*}
    \frac{1}{\|\mathbf{x} - z' \vec{n}_s\|} &=& \frac{1}{\sqrt{r^2 + z'^2 - 2 z' r \cos
    \theta}} = \frac{1}{r} \left(1 - 2 \frac{z'}{r} \cos \theta + \frac{z'^2}{r^2}
    \right)^{-\frac{1}{2}} \\
    &\approx& \frac{1}{r} \left( 1 +  \frac{z'}{r} \cos\theta \right) .
\end{eqnarray*}
Then
\begin{eqnarray*}
     \int_0^\infty\frac{e^{-\redextEddington z'} e^{-\transmission\|\mathbf{x}
    - z' \vec{n}_s\| }}{\|\mathbf{x} - z' \vec{n}_s\|} dz' & \approx &
    \frac{e^{-\transmission r}}{r}  \int_0^\infty
    e^{-(\redextEddington - \transmission \cos \theta )z'}\left(1 + \frac{z'}{r}
    \cos \theta \right) dz'  \\
    &=& \frac{e^{-\transmission r}}{r} \left( \frac{1}{\redextEddington - \transmission
    \cos \theta } + \frac{\cos \theta}{r} \frac{1}{(\redextEddington - \transmission \cos \theta)^2}
    \right).
\end{eqnarray*}

In a highly scattering medium, $\absorption \ll \redscaEddington$. For $g \ne 1$ and $\absorption \ll \transmission \ll \redscaEddington$, this will imply
\begin{equation*}
    \frac{1}{\redextEddington - \transmission \cos \theta} =
    \frac{1}{\redscaEddington} \left(1 +
    \frac{\absorption - \transmission \cos \theta}{\redscaEddington}  \right)^{-1} \approx
    \frac{1}{\redscaEddington}  \left(1 -
    \frac{\absorption - \transmission \cos \theta}{\redscaEddington}  \right)
\end{equation*}
and
\begin{equation*}
    \frac{1}{(\redextEddington - \transmission \cos \theta)^2} =
    \frac{1}{\redscaEddington^2} \left(1 + \frac{\absorption - \transmission \cos \theta}{\redscaEddington}  \right)^{-2} \approx
    \frac{1}{\redscaEddington^2}  \left(1 - 2 \frac{\absorption - \transmission \cos \theta}{\redscaEddington}  \right).
\end{equation*}
The integral can now be approximated by
\begin{eqnarray*}
     \int_0^\infty  \frac{e^{-\redextEddington z'} e^{-\transmission\|\mathbf{x}
    - z' \vec{n}_s\| }}{\|\mathbf{x} - z' \vec{n}_s\|} dz' & \approx &
   \frac{e^{-\transmission r}}{r} \left( \frac{1}{\redscaEddington + \absorption - \transmission
    \cos \theta } + \frac{\cos \theta}{r} \frac{1}{(\redscaEddington + \absorption - \transmission \cos \theta)^2}
    \right) \\
    &\approx & \frac{e^{-\transmission r}}{\redscaEddington r} \left(1 -
    \frac{\absorption - \transmission \cos \theta}{\redscaEddington}      +  \frac{\cos \theta}{\redscaEddington r} \left(1 - 2 \frac{\absorption - \transmission \cos \theta}{\redscaEddington}  \right)
    \right) \\
    &=&\frac{e^{-\transmission r}}{\redscaEddington r} \left(1 -
    \frac{\absorption}{\redscaEddington} + \frac{\cos \theta}{\redscaEddington} \left(
    \transmission + \frac{1}{r} - 2 \frac{\absorption}{\redscaEddington r} \right) + 2
    \frac{\transmission \cos^2\theta}{\redscaEddington^2 r} \right).
\end{eqnarray*}
Inserting this expression in Eq.~\ref{eq:phi},
\begin{eqnarray*}
    \frac{\phi(\mathbf{x})}{\Phi_i} &\approx& - \frac{3 \tilde{g} \redscaEddington e^{- \transmission r}}{4\pi r}
      \\
     && + \frac{3 \left(\redscaEddington + (\tilde{g} +1) \absorption \right) e^{-\transmission r} }{4\pi r}
  \left(1 -
    \frac{\absorption}{\redscaEddington} + \frac{\cos \theta}{\redscaEddington} \left(
    \transmission + \frac{1}{r} - 2 \frac{\absorption}{\redscaEddington r} \right) + 2
    \frac{\transmission \cos^2\theta}{\redscaEddington^2 r} \right) \\
    &=& \frac{3 e^{- \transmission r}}{4\pi r} \left( \left(\redscaEddington +
    (\tilde{g} +1) \absorption \right)  \left( 1 - \frac{\absorption}{\redscaEddington}
    \right) - \tilde{g} \redscaEddington  \right) \\
    && + \frac{3 e^{- \transmission r}}{4\pi r} \cos \theta  \left(1 +
    (\tilde{g} +1) \frac{\absorption}{\redscaEddington} \right) \left( \transmission +
    \frac{1}{r}\left(1 - 2\frac{\absorption}{\redscaEddington}\right) \right) \\
    && + \frac{3 \left(\redscaEddington + (\tilde{g} +1) \absorption \right) e^{-\transmission r} }{4\pi r}
 2 \frac{\transmission \cos^2\theta}{\redscaEddington^2 r} .
\end{eqnarray*}
Now we neglect terms $\absorption/\redscaEddington$ compared to unity
\begin{equation*}
   \frac{\phi(\mathbf{x})}{\Phi_i} \approx  \frac{3 e^{- \transmission r}}{4\pi r} \redscaEddington
    \left( 1 - \tilde{g} \right)
    + \frac{3 e^{- \transmission r}}{4\pi r} \cos \theta \left( \transmission + \frac{1}{r} \right)
    + \frac{3 e^{-\transmission r} }{4\pi r} 2 \frac{\transmission
    \cos^2\theta}{\redscaEddington r}.
\end{equation*}
Introducing the reduced scattering coefficient $\redsca = \redscaEddington ( 1 - \tilde{g}) = \scattering ( 1 - g)$, and neglecting terms $\transmission/\redscaEddington \cdot 1/(\redscaEddington r)$ compared to unity,
\begin{equation*}
    \phi(\mathbf{x}) \approx  \frac{3 \Phi_i e^{- \transmission r}}{4\pi r} \redsca
    + \frac{3 \Phi_i e^{- \transmission r}}{4\pi r} \cos \theta \left( \transmission +
    \frac{1}{r} \right) = \frac{\Phi_i}{4\pi D} \frac{e^{- \transmission r}}{ r} \left( 1
    + 3 D \frac{1 + \transmission r}{r} \cos \theta \right),
\end{equation*}
where $1/D$ is used in place of $3\redsca$, since the $\redsca$ can be approximated by $\redext$ when absorption is negligible compared to scattering. This is a valid assumption in highly scattering media.

\section{Fresnel integrals}
To aid in our calculations, we define the Fresnel transmittance integrals of the first two orders:
$$
\cphi(\eta) = \frac{1}{4\pi}\int_{2\pi} T_{21}(\eta,\theta_o)\cos \theta_o \de \omegavec_o
$$
$$
\cE(\eta) = \frac{3}{4\pi}\int_{2\pi} T_{21}(\eta,\theta_o)\cos^2 \theta_o \de \omegavec_o \, ,
$$
where $\cos\theta_o = \vec{n}_o \cdot \omegavec_o$, and the integral is on the whole hemisphere where $\vec{n}_o \cdot \omegavec_o > 0$. These integrals are similar to $F_{dr}$, but based on Fresnel transmittance instead of Fresnel reflectance.

\section{BSSRDF theory}
The BSSRDF is defined as the ratio of an element of emergent radiance $L_r$ to an element of incident flux $\Phi_i$~\cite{nicodemus77}:
$$
S(\mathbf{x}_i, \omegavec_i, \mathbf{x}_o, \omegavec_o) = \frac{\de L_r(\mathbf{x}_o, \omegavec_o)}{\de \Phi_i(\mathbf{x}_i, \omegavec_i)} \, .
$$
There are various approximations available for the BSSRDF. For the directional dipole~\cite{frisvad14directional}, the BSSRDF is split into the following terms:
$$
S = T_{12} (S_{\delta E} + S_d) T_{21} \, .
$$
Let us consider only the diffusive part, $S_d$. The emergent radiance due to diffusion events is given by
$$
L_{r,d}(\mathbf{x}_o, \omegavec_o) = \eta^2 T_{21} L_{d}(\mathbf{x}_o, \omegavec_{21}) \, ,
$$
where $\omegavec_{21}$ is the refracted vector corresponding to $\omegavec_{o}$:
$$
\omegavec_{21} = \frac{\omegavec_o}{\eta} - \left(\frac{\vec{n}_o \cdot \omegavec_o}{\eta} - \sqrt{1 - \frac{1 - (\vec{n}_o \cdot \omegavec_o
)^2}{\eta^2}}\right) \vec{n}_o \, .
$$
Combining the above equations we obtain:
$$
S_d(\mathbf{x}_i, \omegavec_i, \mathbf{x}_o, \omegavec_o) =  T_{12} S_d T_{21} = \frac{\de L_{r,d}(\mathbf{x}_o, \omegavec_o)}{\de \Phi_i(\mathbf{x}_i, \omegavec_i)} = \eta^2 \frac{\de T_{21}  L_{d}(\mathbf{x}_o, \omegavec_{21}) }{\de \Phi_i(\mathbf{x}_i, \omegavec_i)}
$$
We now integrate over the cosine-weighted hemisphere with $\vec{n}_o \cdot \omegavec_o > 0$ on both sides of the equation:
$$
\int_{2\pi} T_{12} S_d T_{21} \cos \theta_o \de \omegavec_o =  \int_{2\pi} \eta^2 \frac{\de T_{21}  L_{d}(\mathbf{x}_o, \omegavec_{21}) }{\de \Phi_i(\mathbf{x}_i, \omegavec_i)} \cos \theta_o \de \omegavec_o \, .
$$
Assuming no dependency on the outgoing direction, $S_d(\mathbf{x}_i, \omegavec_i, \mathbf{x}_o, \omegavec_o) = S_d(\mathbf{x}_i, \omegavec_i, \mathbf{x}_o)$, and we have
$$
 T_{12} S_d(\mathbf{x}_i, \omegavec_i, \mathbf{x}_o) \int_{2\pi} T_{21} \cos \theta_o \de \omegavec_o =   \eta^2 \frac{\de \int_{2\pi} T_{21}  L_{d}(\mathbf{x}_o, \omegavec_{21}) \cos \theta_o \de \omegavec_o}{\de \Phi_i(\mathbf{x}_i, \omegavec_i)}
$$
$$
 T_{12} S_d(\mathbf{x}_i, \omegavec_i, \mathbf{x}_o) 4\pi\frac{\cphi(\eta)}{\eta^2} =   \frac{\de M_d(\mathbf{x}_o)}{\de \Phi_i(\mathbf{x}_i, \omegavec_i)}
$$
\begin{equation} \label{eq:bssrdf}
 T_{12} S_d(\mathbf{x}_i, \omegavec_i, \mathbf{x}_o) 4\pi\cphi(1/\eta) =   \frac{\de M_d(\mathbf{x}_o)}{\de \Phi_i(\mathbf{x}_i, \omegavec_i)} \, .
\end{equation}
Let us calculate the diffuse radiant exitance first. We insert the diffusion approximation in place of $L_d$ to find
$$
M_d(\mathbf{x}_o) = \int_{2\pi} T_{21}  L_{d}(\mathbf{x}_o, \omegavec_{21}) \cos \theta_o \de \omegavec_o = \int_{2\pi} T_{21}  \left(\frac{\phi(\mathbf{x}_o)}{4\pi} - \frac{3}{4\pi} D\ \omegavec_{21} \cdot \nabla \phi(\mathbf{x}_o)\right) \cos \theta_o \de \omegavec_o
$$
$$
= \underbrace{\frac{\phi(\mathbf{x}_o)}{4\pi} \int_{2\pi} T_{21} \cos \theta_o \de \omegavec_o}_{I_\phi}  - \underbrace{\frac{3}{4\pi} D \int_{2\pi}  \omegavec_{21} \cdot \nabla \phi(\mathbf{x}_o) \cos \theta_o \de \omegavec_o }_{I_{\mathbf{E}}} \, .
$$
And, in a straightforward way,
$$
I_\phi = \cphi(\eta) \phi(\mathbf{x}_o) \, .
$$
The second part is more complicated. Without loss of generality, we rotate the reference coordinate system so that $\omegavec_o = (\cos\phi\sin\theta, \sin\phi \sin\theta, \cos\theta)$ and $\vec{n}_o = (0,0,1)$.  Given this reference system, the refracted vector becomes:
$$
\omegavec_{21} = \frac{(\cos\phi\sin\theta, \sin\phi \sin\theta, \cos\theta)}{\eta} - \left(\frac{\cos\theta}{\eta} - \sqrt{1 - \frac{\sin^2\theta}{\eta^2}}\right) (0,0,1)
$$
$$
= \left(\frac{\cos\phi\sin\theta}{\eta}, \frac{\sin\phi \sin\theta}{\eta},  \sqrt{1 - \frac{\sin^2\theta}{\eta^2}}\right)
$$
When inserted in the dot product in the expression for $I_{\mathbf{E}}$, we get a sum of three components. The first two terms of this sum are zero, since we can first integrate over $\phi$:
$$
\int_0^{2\pi} \cos\phi \de \phi = \int_0^{2\pi} \sin\phi \de \phi = 0 \, .
$$
So we are left only with the cumbersome $z$ term:
$$
I_{\mathbf{E}} = \frac{3}{4\pi} D \frac{\partial \phi_z(\mathbf{x}_o)}{\partial z} \int_{0}^{2\pi}\int_{0}^{\frac{\pi}{2}} T_{21}(\eta,\theta_o) \sqrt{1 - \frac{\sin^2\theta_o}{\eta^2}} \cos \theta_o \sin \theta_o \de \theta_o \de \phi \, ,
$$
where we note that $\frac{\partial \phi_z}{\partial z} = \vec{n}_o\cdot\nabla\phi$. We assume $\eta > 1$, so that the argument of the square root is never negative.  We now perform a substitution using the law of refraction, $\sin\theta_i = \eta \sin\theta_o$. We obtain the following identities:
$$
\de \theta_o = \frac{\eta \cos\theta_i}{\cos\theta_o} \de \theta_i,\ \ \ T_{21}(\eta,\theta_o) = T_{21}(1/\eta,\theta_i),\ \ \ \sqrt{1 - \frac{\sin^2\theta_o}{\eta^2}} = \cos\theta_i.
$$
If we introduce a critical angle $\theta_c = \arcsin(1/\eta)$, we get
$$
I_{\mathbf{E}} = \frac{3}{4\pi} D \vec{n}_o\cdot\nabla\phi(\mathbf{x}_o) \int_{0}^{2\pi}\int_{0}^{\theta_c} T_{21}(1/\eta,\theta_i) \cos\theta_i \cos \theta_o \eta \sin \theta_i \frac{\eta \cos\theta_i}{\cos\theta_o} \de \theta_i \de \phi
$$
$$
= D \vec{n}_o\cdot\nabla\phi(\mathbf{x}_o) \frac{3}{4\pi} \eta^2 \int_{0}^{2\pi}\int_{0}^{\theta_c} T_{21}(1/\eta,\theta_i) \cos^2\theta_i \sin \theta_i \de \theta_i \de \phi
$$
$$
= D \vec{n}_o\cdot\nabla\phi(\mathbf{x}_o) \eta^2 \cE(1/\eta) = \cE(\eta) D \vec{n}_o\cdot\nabla\phi(\mathbf{x}_o) \, .
$$
We can then get our final expression for  $M_d(\mathbf{x}_o)$
$$
M_d(\mathbf{x}_o) =  \cphi(\eta) \phi(\mathbf{x}_o) - \cE(\eta) D \vec{n}_o\cdot\nabla\phi(\mathbf{x}_o) \, .
$$
Inserting into the expression (\ref{eq:bssrdf}), we derive the monopole BSSRDF:
$$
S_d(\mathbf{x}_i, \omegavec_i, \mathbf{x}_o)  =  \frac{1}{4\pi T_{12}\cphi(1/\eta)} \frac{\de M_d(\mathbf{x}_o)}{\de \Phi_i(\mathbf{x}_i, \omegavec_i)}
$$

\subsection{Diffuse monopole BSSRDF}

Using the monopole solution (\ref{eq:monopoint}) for a ray normally incident on an isotropic half-space:
$$
\xfunc{\phi} = \alpha' \frac{\Phi_i}{4 \pi D} \frac{e^{-\transmission r}}{r} \, ,
$$
we can derive the gradient:
$$
\nabla \xfunc{\phi} =  -\alpha'\frac{\Phi_i}{4 \pi D} \frac{e^{-\transmission r}}{r^3} (1 + \transmission r) \mathbf{x} \, .
$$
Thus, we can find the monopole BSSRDF for this configuration:
$$
S_d(\mathbf{x}_i, \mathbf{x}_o)  = \frac{1}{4\cphi(1/\eta)} \frac{\alpha'}{4 \pi^2} \frac{e^{-\transmission r}}{r^3} \left[ \cphi(\eta) \frac{r^2}{D} + \cE(\eta) (1 + \transmission r) (\mathbf{x}_o - \mathbf{x}_i) \cdot \vec{n}_o \right] \, ,
$$
where we used $\Phi_i = T_{12} L_i(\mathbf{x}_i, \omegavec_i)$. This is the monopole version of the better dipole from Eugene d'Eon~\cite{deon12}. To get a monopole version of the standard dipole~\cite{farrell92,jensen01}, we further approximate the above expression using $\cphi(1/\eta) \approx 1/4$, $\cphi(\eta) \approx 0$, and $\cE(\eta) \approx 1$:
$$
S_d(\mathbf{x}_i, \mathbf{x}_o)  =  \frac{\alpha'}{4 \pi^2} \frac{e^{-\transmission r}}{r^3} (1 + \transmission r) \mathbf{x} \cdot \vec{n}_o
$$
This is the monopole version of the standard dipole.

\subsection{Directional monopole BSSRDF}

For the directional dipole, we have the following form:
$$
\xfunc{\phi} = \frac{\Phi}{4\pi D} \frac{e^{- \transmission r}}{ r} \left( 1
    + 3 D \frac{1 + \transmission r}{r} \cos \theta \right).
$$
It is more convenient to do the gradient in spherical coordinates:
$$
\nabla \xfunc{\phi} = \frac{\partial}{\partial r} \xfunc{\phi} \vec{e}_r + \frac{1}{r} \frac{\partial}{\partial \theta} \xfunc{\phi} \vec{e}_\theta +  \frac{1}{r \sin \theta}  \frac{\partial}{\partial \phi} \xfunc{\phi} \vec{e}_\phi \, ,
$$
where
$$
 \frac{\partial}{\partial r} \xfunc{\phi} = - \frac{\Phi}{4\pi D} \frac{e^{- \transmission r}}{ r^2} \left( 3D \frac{2 (1 + \transmission r)  + (\transmission r)^2}{r} \cos\theta + (1 + \transmission r) \right)
$$
and
$$
 \frac{\partial}{\partial \theta} \xfunc{\phi} = - \frac{\Phi}{4\pi D} \frac{e^{- \transmission r}}{ r^2} 3D (1 + \transmission r) \sin\theta \, .
$$
Finally, $\frac{\partial}{\partial \phi} \xfunc{\phi} = 0$. Given our choice of basis for derivation, we can use the following identities:
$$
\vec{e}_r = \frac{\mathbf{x}}{r}
$$
$$
- \vec{e}_\theta \sin\theta = \omegavec_{12} - \vec{e}_r \cos\theta \, .
$$
Inserting the identities, we get an expression for the gradient:
\begin{multline*}
\nabla \xfunc{\phi} = - \frac{\Phi}{4\pi D} \frac{e^{- \transmission r}}{ r^2} \left( 3D \frac{2 (1 + \transmission r)  + (\transmission r)^2}{r} \cos\theta + (1 + \transmission r) \right) \vec{e}_r \\ + \frac{1}{r}  \frac{\Phi}{4\pi D} \frac{e^{- \transmission r}}{ r^2} 3D (1 + \transmission r) (- \vec{e}_\theta \sin\theta)
\end{multline*}
\begin{multline*}
\nabla \xfunc{\phi} =\frac{\Phi}{4\pi D} \frac{e^{- \transmission r}}{ r^3} \bigg[ \left( -3D \frac{2 (1 + \transmission r)  + (\transmission r)^2}{r} \cos\theta - (1 + \transmission r) \right) r \vec{e}_r  \\-  3D (1 + \transmission r) \cos\theta \vec{e}_r + 3D (1 + \transmission r) \omegavec_{12} ]
\end{multline*}
\begin{multline*}
\nabla \xfunc{\phi} =\frac{\Phi}{4\pi D} \frac{e^{- \transmission r}}{ r^3} \bigg[ -\left( 3D \frac{3 (1 + \transmission r)  + (\transmission r)^2}{r} \cos\theta + (1 + \transmission r) \right) \mathbf{x}  \\ + 3D (1 + \transmission r) \omegavec_{12} ] \, .
\end{multline*}
We can now do the same as above, obtaining the directional monopole BSSRDF with $\mathbf{x} = \mathbf{x}_o - \mathbf{x}_i$ and $r = \|\mathbf{x}\|$:
\begin{multline*}
S_d'(\mathbf{x}_i, \omegavec_i, \mathbf{x}_o)  = \frac{1}{4\cphi(1/\eta)} \frac{1}{4 \pi^2} \frac{e^{-\transmission r}}{r^3} \bigg[ \cphi(\eta) \left(\frac{r^2}{D} +  3 (1 + \transmission r) \mathbf{x}\cdot\omegavec_{12} \right) \\ {}-\cE(\eta) \left(3D (1 + \transmission r) \omegavec_{12} \cdot \vec{n}_o - \left((1 + \transmission r) + 3D \frac{3 (1 + \transmission r)  + (\transmission r)^2}{r^2}\mathbf{x}\cdot\omegavec_{12}\right) \mathbf{x} \cdot \vec{n}_o\right) \bigg] \, .
\end{multline*}
%Alternative form, using $t = 1 + \transmission r$:
%\begin{multline*}
%S_d'(\mathbf{x}_i, \omegavec_i, \mathbf{x}_o)  = \frac{1}{4\cphi(1/\eta)} \frac{\transmission}{4 \pi^2} \frac{e^{1-t}}{(t-1)^3} \bigg[\frac{\cphi}{D}(t-1)^2+ 3 \cphi \transmission  t (t-1) (\vec{x} \cdot  \omegavec_{12}) \\- 3 \cE D  \transmission^2   t (\vec{n}_o \cdot  \omegavec_{12}) + \cE \transmission^2  (t+1) (\vec{x} \cdot \vec{n}_o) + \cE  D \transmission^3  \frac{t^2+t+1}{t-1} (\vec{x} \cdot  \omegavec_{12}) (\vec{x} \cdot \vec{n}_o)\bigg] \, .
%\end{multline*}

\bibliographystyle{plain}

\bibliography{jensen_note}

\end{document} 