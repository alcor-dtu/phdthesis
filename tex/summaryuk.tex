\chapter{Summary}

Interactive rendering applications are becoming more and more prominent in everyday life. In many fields, including manufacturing, product design and entertainment, photorealistic rendering is useful in predicting the appearance of complex materials. However, due to production and time constraints, applications need to be interactive to provide immediate feedback to the user.

In this thesis, we address the challenge of proposing new photorealistic interactive rendering techniques, that leverage the parallel power of graphics processing units (GPUs) in order to effectively create renderings based on the laws of physics. These techniques propose effective caching and filtering schemes in order to efficiently reuse data, either across space or across time.     
 
This thesis offers insight into different areas of computer graphics, including scene reconstruction, material parameter estimation, efficient data structures and physically based rendering models. Our goal is to explore the different compromises and trade-offs that are necessary to achieve accurate photorealistic renderings. More specifically, we contribute with two techniques: the first relates to fast rendering of translucent materials, accounting for directional effects of subsurface scattering. The second technique contributes with a fast reprojection scheme to improve temporal stability in interactive ray tracing, that can be applied on top of existing rendering algorithms. On top of these, we propose an innovative validation pipeline to compare renderings with actual images, with the final purpose of validating existing rendering and reconstruction techniques against a picture of the real world. 

With these contributions, this thesis proves how it is possible to use effective caching schemes to effectively improve existing techniques to handle more complex optical effects, maintaining the time constraints of interactive rendering environments.