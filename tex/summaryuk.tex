\chapter{Summary}

Interactive rendering applications are becoming more and more prominent in everyday life. In many fields, including manufacturing, product design and entertainment, photorealistic rendering is becoming more and more prominent, in order to correctly predict appearance of complex materials. So, in many interactive applications, there is a need for fast interactive techniques that achieve based on physically based principles. 

In this thesis, we address the challenge of proposing new photorealistic interactive rendering techniques, that leverage the parallel power of graphics processing unit (GPUs) in order to effectively create renderings based on physical principles. These techniques propose effective caching and filtering schemes in order to efficiently reuse data, both spatially and temporally.     
 
This thesis offers insight into different areas of computer graphics, including scene reconstruction, material parameter estimation, efficient data structures and physically based rendering models. In addition, we explore the different compromises and trade-offs that are necessary to achieve accurate photorealistic renderings. More specifically, we contribute with two innovative techniques: the first relates to fast rendering of translucent materials and includes directional effects of subsurface scattering into consideration. The second technique contributes with a fast reprojection scheme to improve temporal stability in interactive ray tracing, that can be easily applied on top of existing rendering algorithms. On top of these, we propose an innovative validation pipeline to compare renderings with actual images, with the final purpose of validating existing rendering and reconstruction techniques against a picture of the real world. 

With these contributions, this thesis proves how it is possible to use effective caching schemes to effectively improve existing techniques to handle more complex optical effects, maintaining the time constraints of interactive rendering environments.