\chapter{Summary}

Interactive rendering applications are becoming more and more prominent in every day life. In many fields in manufacturing, product design and entertainment, photorealistic rendering is becoming more and more prominent, in order to correctly predict the appearance of complicated materials. However, in many interactive applications, there is a need for fast interactive techniques that achieve physically based results. 

In this thesis, we address the challenge of proposing new photorealistic interactive rendering techniques, that leverage the parallel power of graphics processing unit (GPUs) in order to effectively create renderings that are physically based. These techniques exploit caching and filtering schemes in order to efficiently reuse data, both spatially and temporally.     
 
This thesis offers insight into different areas of computer graphics, including scene reconstruction, material parameter estimation, efficient data structures and physically based rendering models. In addition, we explore the different compromises and trade-offs that are necessary to achieve accurate photorealistic renderings. More specifically, we contribute with two innovative technique, one related to fast rendering of translucent materials, including directional effects of subsurface scattering into consideration. Our other technique contributes with a fast reprojection scheme to improve temporal stability in interactive ray tracing, that can be easily applied on top of existing algorithms. On top of these, we propose an innovative validation pipeline to compare renderings with actual images, with the final purpose of validating existing rendering and reconstruction techniques. 

With these contributions, this thesis proves how it is possible to use effective caching schemes to effectively improve existing techniques to handle more complex optical effects, maintaining the time constraints of interactive rendering environments.