\chapter{Summary (Danish)}
\begin{otherlanguage}{danish}
Anvendelse af interaktive renderingsprogrammer bliver mere og mere fremtrædende i hverdagslivet. På mange områder, såsom produktfremstilling, produktdesign og underholdningsindustrien, bruges fotorealistisk rendering til forudsigelse af komplekse materialers udseende. Dog er der, pga. produktions- og tidsbegrænsninger, behov for interaktive programmer fordi brugere har brug for umiddelbar feedback.

I denne afhandling adresserer vi omtalte udfordring ved at foreslå nye interaktive fotorealistiske renderingsteknikker som bygger på grafikkortets (GPU’ens) egenskaber i forhold til parallelprocessering og derfor effektivt kan skabe renderinger der bygger på fysikkens love. Disse teknikker indeholder forslag mht. effektive caching- og filtreringsprocedurer til effektiv genbrug af data over rum eller over tid.

Vi giver indsigt i forskellige områder af computergrafik, herunder scene rekonstruktion, materiale parameterestimering, effektive datastrukturer og fysisk baserede renderingsmodeller. Vores mål er at udforske de forskellige kompromiser og afvejninger som er nødvendige for at opnå nøjagtige fotorealistiske renderinger. Mere specifikt bidrager vi med 2 teknikker: Den første er relateret til hurtig rendering af halvgennemsigtige materialer hvor der tages højde for retningsbestemte effekter i lysspredningen under overfladen. Den anden teknik er et bidrag i form af en hurtig genprojiceringsprocedure til forbedring af stabilitet over tid i interaktiv strålesporring (ray tracing). Dette kan anvendes som overbygning på eksisterende renderingsalgoritmer. Derudover bibringer vi også en innovativ valideringspipeline til sammenligning af renderede billeder med egentlige billeder. Hensigten er validering af eksisterende renderings- og rekonstruktionsteknikker op imod et billede af den virkelige verden.

Med disse bidrag demonstrerer vi hvordan det er muligt at anvende effektive caching procedurer til effektivt at forbedre eksisterende teknikker til at kunne håndtere mere komplekse optiske effekter og stadig leve op til tidsbegrænsningerne i et interaktivt renderingsmiljø.
\end{otherlanguage}