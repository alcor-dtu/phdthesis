\chapter{Conclusions}
In this thesis, we have presented the results of the past three year of Ph.D. studies, that started with the goal of developing new techniques for interactive rendering, with an eye onto interactive photorealism. The exploration of different techniques during the Ph.D. studies did not followed a predefined scheme. The individual contributions have been developed as consequence of the findings along the way. We are happy that we managed to keep our contribution over a common theme, bringing photorealistic rendering into the interactive domain.

Over the course of the Ph.D. studies we contributed with a number of results and publications. These results are relevant in many fields, including product visualization, architectural rendering, interactive previews, previews of rendering results, and video game production. We have presented a range of publications that contribute with techniques that can be employed in these fields, addressing various important challenges in each area. 

In recent years, the real time rendering community is pushing more and more towards physically based models. Given this, we can use clever techniques to introduce additional photorealism in existing techniques. We saw a need for more accurate predictive rendering in an industrial domain in Contribution~\ref{sec:juice}, where we tied industrial parameters to the rendering of physically based apple juice. We first investigated the need for photorealism to validate existing reconstruction and acquisition techniques in Contribution~\ref{sec:glass}. Moreover, we proved the need of accurate photorealistic rendering to measure unknown radiometric parameters. 

After this excursion in hyper photorealistic rendering, we moved into developing techniques that improve upon existing algorithms to further enhance physically based models. In our Contribution~\ref{sec:interactivedirsss}, we contributed with the first technique able to render directional BSSRDF models on deformable objects, allowing also to transport emergent scattered light across the scene. Inspired by our industrial lookout, we set on solving the problem of temporally stable global illumination, in the growing field of interactive ray tracing. We contributed with a interactive technique in Contribution~\ref{sec:srt} that allows temporally stable sharp global illumination, and that can be easily added on top of many existing algorithms. Finally, in Contribution~\ref{sec:vrbrdf} we contributed with a proof of concept to the growing field of virtual reality rendering, introducing physically based measured BRDFs in a real time environment with hard constraints.   

All the contributions listed above contributed to achieve new insights in the individual areas of interest. We can summarize the highlights of the single contributions in this thesis as the following: 
\label{sec:conclusion}
\begin{itemize}
\item Investigated the challenges in comparing images with photorealistic renderings (Contributions~\ref{sec:juice}, \ref{sec:glass}).
\item Proposed a reconstruction and assembly pipeline that allows to compare images to renderings of the same scene (Contribution~\ref{sec:glass}).
\item Created a new dataset of transparent objects scene and CT scan data (Contribution~\ref{sec:glass}).
\item Proposed new techniques to estimate material parameters (Contributions~\ref{sec:juice}, \ref{sec:glass}).
\item Explored the challenges of fast interactive photorealistic rendering for rendering and quality assurance (Contributions~\ref{sec:juice},\ref{sec:interactivedirsss}).
\item Demonstrated the first interactive application of light-directional subsurface scattering (Contribution~\ref{sec:interactivedirsss}).
\item Developed a new interactive ray tracing technique to improve temporal stability without sacrificing sharpness (Contribution~\ref{sec:srt}).
\item Created new interactive rendering technques on to improve photorealistic light transport (Contributions~\ref{sec:interactivedirsss}, \ref{sec:srt}).
\item Created a novel Virtual Reality application to showcase phisically based materials in a hard real time context (Contribution~\ref{sec:vrbrdf}).
\end{itemize}

Based on these contributions, we conclude that the goal of developing new interactive techniques that push the boundaries of photorealistic rendering in the interactive and real time domain has been achieved. 