\chapter{Background}
\label{sec:background}
\begin{itemize}
\item Introduction of relevant theory behind the thesis
\item In photorealistic, introduction to most relevant quantities, namely radiometric quantities, brdfs and bssrdfs. Scattering and non scattering media.
\item Offline and real time techniques. GPUs. 
\item ray tracign and rasterization. bleeding between the two.
\end{itemize}

\section{Photorealistic rendering}

\subsection{Radiometry}
Radiometry is that branch of science that measures electromagnetic radiation. We will define the basic radiometric quantities, in order to then introduce the two main reflectance functions of interest in this thesis, namely the BRDF and the BSSRDF. 

First, we consider an emitter of light, or an ideal light source.
The first quantity we define is radiant flux or radiant power $\Phi$, defined in term of Watts $[W]$. This quantity defines the amount of energy per second the 

Definitions
Flux
Irradiance
Intensity 
Radiance
Why is radiance important
Discussion about properties of the quantities

\subsection{The BRDF}
Definition
Reflectance equation
Properties
Analytical model
Data driven models

\subsection{Scattering media}
Scattering equation
Single terms: absorption, in scattering out scattering

\subsection{The BSSRDF}
Definition
Dipole configuration
Diffusion approximation
Example as the standard dipole

\subsection{Rendering techniques}
\fixme{Discuss with Jeppe whether to move in Related work?}
\begin{itemize}
\item The rendering equation
\item Path tracing
\item Rendering with reflectance functions
\end{itemize}

\section{Performance of rendering techniques} 

A comparison about rendering techniques. Offline vs realtime
Gpus vs cpus


\subsection{Offline rendering techniques}
Accuracy, slow rendering times. Focus on performance but mostly on achieving converged frames

\subsection{Real-time rendering techniques}
Disregard for physically based, but new interest in the lastest years

Squeezing as much as possible from the GPU. 



