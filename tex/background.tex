\chapter{Background}
\label{sec:background}

\section{Radiometry}
\begin{figure}
\centering
   \def\svgwidth{0.4\textwidth}
   \input{figures/radiance.pdf_tex} \\
\caption{Configuration to define radiance.} %The red rectangle shows where we estimated RMSE in Table \ref{table:quant}.}
\label{fig:radiance}
\end{figure}
Radiometry is that branch of science that measures electromagnetic radiation. We will define the basic radiometric quantities, in order to then introduce the two main reflectance functions of interest in this thesis, namely the BRDF and the BSSRDF. In particular, we want to give a definition of radiance, the most useful quantity in describing light transport along rays.

First, we consider an ideal point light source in space. The source emits a certain amount of energy $U$, measured in Joules $[J]$. The first quantity we derive is radiant flux or radiant power $\Phi$, defined as the amount of energy per second emitted by the light:
\begin{equation*}
\Phi = \frac{d U}{d t}  \siunit{\watt}.
\end{equation*}
The radiant flux represents the overall power the light emits in all directions overall. We usually want to be more descriptive on how a light or a surface is emitting light, since not all the sources we consider are ideal. First, we are interested on how the light emission changes  We then define $I$ as radiant intensity, i.e. the amount of flux the light emits towards a specific direction $\vec{\omega}$:
\begin{equation*}
I(\vec{\omega}) = \frac{d \Phi}{d \omega}  \siunit{\watt \per \steradian}.
\end{equation*}   
Where $d \omega$ is an infinitesimal solid angle around direction $\vec{\omega}$. An isotropic point light, by definition, as constant intensity across all directions.

We now consider a surface that receives light. In particular, we consider an infinitesimal oriented element of this surface $d \vec{A}$ around a point $\mathbf{x}$. The orientation of the surface is usually called the \emph{normal} and indicated by $\vec{n}$ We can now define irradiance as the amount of incoming flux received per unit area:
\begin{equation*}
E(\mathbf{x}) = \frac{d \Phi}{d A}  \siunit{\watt \per \square \metre}
\end{equation*}
Similarly, we have a dual quantity for the flux \emph{emitted} by a unit element surface. We call this quantity \emph{radiosity} and we define it with the symbol $B$.
The most simple irradiance is the one emitted by an isotropic point light. In this case, the irradiance at a distance $R$ from the point light is 
\begin{equation*}
E = \frac{\Phi}{4 \pi R} \siunit{\watt \per \square \metre}
\end{equation*}
So, the father we go from a point light, the less irradiance we receive, because the power $\Phi$ has to spread across a larger area.

Finally, we combine the definitions above into one, to define the last important basic radiometric quantity, \emph{radiance}:
\begin{equation*}
L(\mathbf{x}, \vec{\omega}) = \frac{d^2 \Phi}{d A \cos\theta d \omega}  \siunit{\watt \per \square \metre \per \steradian}
\end{equation*}
Where $\cos \theta = \vec{n} \cdot \vec{\omega}$ is the cosine of the angle between the normal and the direction of evaluation. $d A \cos\theta$ is also called the \emph{projected area element}. As irradiance, radiance can be incoming or outgoing from a specific point. We indicate these quantities with $L_i$ and $L_o$, respectively. In graphics, radiance is usually the most useful quantity for two main reasons. First, the other quantities can be easily computed from radiance through radiometric integrals:
\begin{figure}
\centering
   \def\svgwidth{0.8\textwidth}
   \input{figures/etendue.pdf_tex} \\
\caption{Configuration to prove the equality of radiance across a ray.} %The red rectangle shows where we estimated RMSE in Table \ref{table:quant}.}
\label{fig:etendue}
\end{figure}
\begin{equation*}
\begin{split}
\Phi &= \int_{\Omega} \int_{A} L(\mathbf{x}, \vec{\omega})  (\vec{n} \cdot \vec{\omega}) \ dA \ d \vec{\omega} \\
I(\vec{\omega}) &= \int_{A} L(\mathbf{x}, \vec{\omega})  (\vec{n} \cdot \vec{\omega}) \ dA  \\
E(\mathbf{x}) &= \int_{\Omega} L(\mathbf{x}, \vec{\omega})  (\vec{n} \cdot \vec{\omega}) \ d \vec{\omega} 
\end{split}
\end{equation*}
Where $\Omega$ and $A$ are the hemisphere around $\vec{n}$ and the total surface, respectively. Second, radiance carried by a ray \emph{in vacuo} is constant. We can prove this quite easily, with the aid of Figure~\ref{fig:etendue}. Note that for point $\mathbf{x}_o$, we have $d \omega_o = \frac{d A_i \cos\theta_i}{r^2}$, and dually the same for $\mathbf{x}_i$. Then:
\begin{equation*}
\begin{split}
L_i(\mathbf{x}_i, \vec{\omega}_i) = \frac{d^2 \Phi}{d A_i \cos\theta_i d \omega_i} &= 
\frac{d^2 \Phi}{d A_i \cos\theta_i \frac{d A_o \cos\theta_o}{r^2}}  
\\ &= \frac{d^2 \Phi}{\frac{d A_i \cos\theta_i}{r^2} d A_o \cos\theta_o } 
= \frac{d^2 \Phi}{d A_o \cos\theta_o d \omega_o} = L_o(\mathbf{x}_o, \vec{\omega}_o)
\end{split}
\end{equation*}

Note that we assume that the flux does not varies across the path, that is generally true if no objects are in the way and there is not medium in between the two points causing scattering events. 

\subsection{The BSSRDF}

\begin{figure}
\centering
   \def\svgwidth{0.4\textwidth}
   \input{figures/bssrdf_geometry.pdf_tex} \\
\caption{Configuration in which we define the BSSRDF for a generic surface.} %The red rectangle shows where we estimated RMSE in Table \ref{table:quant}.}
\label{fig:bssrdf_configuration}
\end{figure}


So far, we have defined radiometric quantities, without worrying about the interaction at the surface. We will now describe how this radiometric quantities change once they encounter a surface. Let us first consider a surface illuminated by a light. Let us consider the portion of the flux $d \Phi_i$ arriving from direction $\vec{\omega}_i$ on a surface element $d A_i$ centered on a point $\mathbf{x}_i$. Due to surface interaction, part of the incoming light will emerge on a point $\mathbf{x}_o$, in direction $\vec{\omega}_o$. We consider the proportionality factor between the emitted radiance and the incoming flux:
\begin{equation}
\label{eq:bssrdf}
d L(\mathbf{x}_i, \vec{\omega}_i, \mathbf{x}_o, \vec{\omega}_o) = S(\mathbf{x}_i, \vec{\omega}_i, \mathbf{x}_o, \vec{\omega}_o) d \Phi_i(\mathbf{x}_i, \vec{\omega}_i)  \siunit{\watt \per \square \metre \per \steradian}
\end{equation}
The factor $S$, dependent on the surface materials, is called \emph{Bidirectional scattering-surface reflectance distribution function} (BSSRDF). The units for the BSSRDF are $\siunitnospace{\per \square \metre \per \steradian}$. From the definition of BSSRDF and flux, we obtain the extended form of the rendering equation:
\begin{equation}
\label{eq:bssrdfintegral}
L(\mathbf{x}_o, \vec{\omega}_o) = \int_A \int_\Omega S(\mathbf{x}_i, \vec{\omega}_i, \mathbf{x}_o, \vec{\omega}_o) L_i(\mathbf{x}_i, \vec{\omega}_i) (\vec{n} \cdot \vec{\omega}_i) d A_i d \vec{\omega}_i  \siunit{\watt \per \square \metre \per \steradian}
\end{equation}
Note that we did not assume anything about the material, deriving the BSSRDF from purely radiometric quantities. 
Which is the standard form of the rendering equation. To get the complete outgoing radiance distribution, we sum over the radiance emitted by the body:
\begin{equation*}
L_o(\mathbf{x}, \vec{\omega}) = L_e(\mathbf{x}, \vec{\omega}) + L(\mathbf{x}, \vec{\omega}) 
\end{equation*}

\subsection{Local solution: the radiative transfer equation}
\begin{figure}
\centering
\begin{tabular}{@{}c@{}c@{}}
\def\svgwidth{0.45\textwidth}\input{figures/rte_absorption.pdf_tex} & 	 \def\svgwidth{0.45\textwidth}\input{figures/rte_emission.pdf_tex} \\
Absorption & Emission \\[1em]
\def\svgwidth{0.45\textwidth}\input{figures/rte_inscatter.pdf_tex} & 	 	 \def\svgwidth{0.45\textwidth}\input{figures/rte_outscatter.pdf_tex} \\
In-scattering &  Out-scattering \\
\end{tabular}
\caption{Individual processes in the local form of the radiative transfer equation: absorption, emission, in-scattering and out-scattering. The directional derivative $\vec{\omega} \cdot \nabla L(\mathbf{x}, \vec{\omega})$ corresponds on the variation over an infinitesimal length element $\text{d}r$ in direction $\vec{\omega}$. } 
\label{fig:rte_elements}
\end{figure}
Of the two solutions on how light propagates into interface bounded volumes, we start with the local solution, i.e. the radiative transfer equation. This is a integro-differential equation that describes the behaviour of radiance $L(\mathbf{x}, \vec{\omega})$ at a point $\mathbf{x}$ in a medium towards direction $\vec{\omega}$. We assume a system that respects linear optics (excluding i.e. flourescent materials), and in the steady state (i.e. the radiance $L$ changes in time at speeds not comparable to the speed of light, $\frac{dL(\mathbf{x}, \vec{\omega})}{c dt} \approx 0$).

Traditionally, light travelling a medium is subject to four processed: absorption, emission, inscattering and outscattering. Each of these effects can be described on how it makes radiance change along the unit direction $\vec{\omega}$. This can be matematically expressed as the directional derivative $\vec{\omega} \cdot \nabla L(\mathbf{x}, \vec{\omega})$, expressed in $\siunitnospace{\watt \per \cubic \metre \per \steradian}$. 

\textbf{Emission} increases the overall radiance of the ray by a term $q(\mathbf{x}, \vec{\omega})$:
\begin{equation*}
\vec{\omega} \cdot \nabla L(\mathbf{x}, \vec{\omega}) = q(\mathbf{x}, \vec{\omega}) \siunit{\watt \per \cubic \metre \per \steradian}
\end{equation*} 

\textbf{Absorption} is the dual effect of emission, where photons are absorbed by the material, and generally being transformed into heat. A probability distribution, called absorption cross section $\sigma_a(\mathbf{x}, \vec{\omega})  \siunitnospace{\per \metre} $, describes the probability that a photon is absorbed per unit travelled within the medium. In formulas,
\begin{equation*}
\vec{\omega} \cdot \nabla L(\mathbf{x}, \vec{\omega}) = - \sigma_a(\mathbf{x}, \vec{\omega}) L(\mathbf{x}, \vec{\omega})  \siunit{\watt \per \cubic \metre \per \steradian}
 \end{equation*}

\textbf{Out-scattering} describes photons that are deflected from their original path $\vec{\omega}$  due to interaction with the atoms of the material. The process is similar to absorption, with a different coefficient named scattering cross section $\sigma_s(\mathbf{x}, \vec{\omega}) \siunitnospace{\per \metre}$. This gives a similar radiance loss as absorption:
\begin{equation*}
\vec{\omega} \cdot \nabla L(\mathbf{x}, \vec{\omega}) = - \sigma_s(\mathbf{x}, \vec{\omega}) L(\mathbf{x}, \vec{\omega})
 \siunit{\watt \per \cubic \metre \per \steradian}
\end{equation*}
Absorption and out-scattering, given their similarities, are often combined into an unique effect, \emph{attenuation}, described by a unique value called the extinction coefficient $\sigma_t(\mathbf{x}, \vec{\omega}) =\sigma_s(\mathbf{x}, \vec{\omega}) + \sigma_a(\mathbf{x}, \vec{\omega}) \siunitnospace{\per \metre}$.

\textbf{In-scattering}, finally, describes the radiance aligning towards $\vec{\omega}$ from other scattering events. Let us consider another direction $\vec{\omega}'$ from $\mathbf{x}$. We define a probability distribution for a photon to scatter from $\vec{\omega}'$ towards an infinitesimal solid angle $d{\omega}$ around $\vec{\omega}$. This probability distribution is called the \emph{phase function}  $p(\mathbf{x}, \vec{\omega}', \vec{\omega}) \siunitnospace{ \per \steradian}$. From the phase function, we can calculate the dimensionless radiometric property $g(\mathbf{x}, \vec{\omega})$, describing the mean of the cosine of the angle between $\vec{\omega}$ and $\vec{\omega}'$:
\begin{equation*}
g(\mathbf{x}, \vec{\omega}) = \int_{4\pi} (\vec{\omega}' \cdot \vec{\omega}) p(\mathbf{x}, \vec{\omega}', \vec{\omega}) d\vec{\omega}'
\siunit{-}
\end{equation*}
With the phase function, and integrating over all directions, we get the effect of in-scattering.
\begin{equation*}
\vec{\omega} \cdot \nabla L(\mathbf{x}, \vec{\omega}) = \sigma_s(\mathbf{x}, \vec{\omega}) \int_{4\pi} L(\mathbf{x}, \vec{\omega}')  p(\mathbf{x}, \vec{\omega}', \vec{\omega}) d \vec{\omega}'\siunit{\watt \per \cubic \metre \per \steradian}
\end{equation*}
Note the multiplication by $\sigma_s$ to include the fact that photons will in-scatter with probability $\sigma_s$. 

We can now combine all terms to obtain the complete form of the radiative transfer equation:
\begin{equation}
\label{eq:rte}
\vec{\omega} \cdot \nabla L(\mathbf{x}, \vec{\omega}) = q(\mathbf{x}, \vec{\omega}) - \sigma_t(\mathbf{x}, \vec{\omega}) L(\mathbf{x}, \vec{\omega}) + \sigma_s(\mathbf{x}, \vec{\omega}) \int_{4\pi} L(\mathbf{x}, \vec{\omega}')  p(\mathbf{x}, \vec{\omega}', \vec{\omega}) d \vec{\omega}'
\end{equation}
\subsection{Connecting BSSRDFs and the radiative transfer equation}

Given the above definition of BSSRDF (Equation~\ref{eq:bssrdf}), one may wonder on how the BSSRDF is related to any scattering process at all, since it simply describe a relationship between an incoming and one outgoing radiometric quantity. In this section, starting from the definition of BSSRDF, we will derive the local integro-differential form of the radiative transfer Equation~\ref{eq:rte}, to show how the BSSRDF fits nicely as a mathematical description of an underlying scattering process. 

To show this relationship, we need some additional mathematical construct to simplify notation, namely \emph{functionals}. A functional in this case is a operator that associates a function to another function. For example, let us define the functional $\operator{Q}$ to integrate over an hemisphere:
\begin{equation*}
\operator{Q} = \int_\Omega [\ \ ] d\omega.
\end{equation*}
So, if we for example write $E = L\operator{Q}$, it corresponds to the equation
\begin{equation*}
E(\mathbf{x}) = \int_\Omega L(\mathbf{x}, \vec{\omega}) d\omega.
\end{equation*}
When obvious, we will drop the dependencies on $\mathbf{x}$ and $\omega$ from the functional form for clarity's sake. We can now define a new functional operator $\opa$, also calle the \emph{standard operator}:
\begin{equation*}
\opa(a,b) = \int_a \int_{\Omega_i^+} [\ \ ] S(\mathbf{x}_i, \vec{\omega}_i, \mathbf{x}_o, \vec{\omega}_o) (\vec{n}_i \cdot \vec{\omega_i}) d \omega_i d A_i
\end{equation*}
Where $a$ and $b$ are parts of the surface of the medium containing $\mathbf{x}_i$ and $\mathbf{x}_o$, respectively. $\Omega_i^+$ is the hemisphere oriented towards $\vec{n}_i$. This allows us to write $L = L_i \opa(a,b)$ for equation~\ref{eq:bssrdfintegral}, greatly simplifying notation.

\begin{figure}
\centering
   \def\svgwidth{0.8\textwidth}
   \input{figures/cylinder_geometry.pdf_tex} \\
\caption{Cylinder configuration to prove the radiance solution.} %The red rectangle shows where we estimated RMSE in Table \ref{table:quant}.}
\label{fig:cylinder}
\end{figure}

Let us consider the configuration of Figure~\ref{fig:cylinder}. In this figure, we have a path $P_r(\mathbf{x}_i,\vec{\omega})$ starting at $\mathbf{x}_i$, with direction $\vec{\omega}$ for $r$ units, terminating in $\mathbf{x}_o$ (so that $\mathbf{x}_o = \mathbf{x}_i + r \vec{\omega}$). Let us now consider a cylindrical medium $C$, composed of three parts: a top circle $a$ on $\mathbf{x}_i$, a bottom circle $b$ on $\mathbf{x}_o$ and a flank surface $c$. We orient the cylinder towards $\vec{n}_i = -\vec{\omega}$. This defines a direction for the hemispheres, e.g. $\Omega_i^+$ and $\Omega_i^-$ are the hemispheres centered in $\mathbf{x}_i$  and oriented towards or against $\vec{n}_i$, respectively. As final bit of notation, we indicate with $L^+(a)$ some radiance $L(\mathbf{x}, \vec{\omega})$ where $\mathbf{x} \in a$ and $\vec{\omega} \in \Omega^+$.

At the steady state, the total radiance going out of the cylinder through the surface $b$ is:
\begin{equation}
\label{eq:steadystate}
L^-(b) = L^+(b)\opa(b,b) + L^-(a)\opa(a,b)  + L^-(c)\opa(c,b)
\end{equation}
Intuitively, the first term on the right of the summation represent the radiance not going out due to reflection, while the second and third term represent the radiance entering in $a$ and $c$ scattering out of $C$ through $b$, respectively. We want now to find the behavior of the system at equilibrium, when the cylinder becomes thinner and thinner ($C\rightarrow P_r(\mathbf{x}_i,\vec{\omega})$). We will analyze each one of the terms in Equation~\ref{eq:steadystate} independently. 

The first term of Equation~\ref{eq:steadystate} can be shown to tend to zero as the cylinder shrinks. This comes from the fact that the reflectance of a transparent plane is zero, as all the radiance contribution passes through. In formulas
\begin{equation*}
\lim_{C\rightarrow P_r(\mathbf{x}_i,\vec{\omega})} [L^+(b)\opa(b,b)] (\mathbf{x}_o, \vec{\omega}) = 0.
\end{equation*}
The second term defines the radiance transmitted inbetween $a$ and $b$, as the cylinder shrink, the photons have less and less room to scatter within the cylinder, so only the photons staying directly on $\vec{\omega}$ (even if they scatter back and forth) will be considered at the end. Let us consider the limit:
\begin{equation*}
L_r^0(\mathbf{x}_o, \vec{\omega}) = \lim_{C\rightarrow P_r(\mathbf{x}_i,\vec{\omega})} [L^-(a)\opa(a,b)] (\mathbf{x}_o, \vec{\omega})
\end{equation*}
At the limit $a \rightarrow \mathbf{x}_i$, we have $L^-(a) = L^0(\mathbf{x}_i, \vec{\omega})$, that can be brought out: 
\begin{equation*}
L_r^0(\mathbf{x}_o, \vec{\omega}) = L^0(\mathbf{x}_i, \vec{\omega})\lim_{C\rightarrow P_r(\mathbf{x}_i,\vec{\omega})} [\opa(a,b)] (\mathbf{x}_o, \vec{\omega}) = L^0(\mathbf{x}_i, \vec{\omega}) T_r(\mathbf{x}_i, \vec{\omega})
\end{equation*}
Where the last term is called \emph{beam transmittance}, i.e. the amount of light blocked on the direct path. A dual term $1 -  T_r(\mathbf{x}_i, \vec{\omega})$ can be defined, called \emph{beam attenuation}. The rate of change of beam attenuation per unit length at $\mathbf{x}_i$ on $\vec{\omega}$ is called \emph{extinction coefficient}:
\begin{equation*}
\sigma_t(\mathbf{x}_i, \vec{\omega}) = \lim_{r\rightarrow 0} \frac{1 - T_r(\mathbf{x}_i, \vec{\omega})}{r}
\end{equation*}
Which leads to a natural definition for the beam transmittance:
\begin{equation*}
T_r(\mathbf{x}_i, \vec{\omega}) = \exp\left(-\int_0^r \sigma_t(\mathbf{x}_i + r' \vec{\omega}, \vec{\omega}) dr'\right)
\end{equation*}
We now miss to calculate the last term in the summation, that we call $L^*$:
\begin{equation*}
L_r^*(\mathbf{x}_o, \vec{\omega}) = \lim_{C\rightarrow P_r(\mathbf{x}_i,\vec{\omega})} [L^-(c)\opa(c,b)] (\mathbf{x}_o, \vec{\omega})
\end{equation*}
We won't include the full derivation, but by subdividing the cylinder into infinitesimal tiny slices, each with its own transmittance, and then taking the limit, to obtain a integral form for $L_r^*$:
\begin{equation*}
L_r^*(\mathbf{x}, \vec{\omega}) = \int_0^r L_*(\mathbf{x}', \vec{\omega}) T_{r-r'}(\mathbf{x}', \vec{\omega})  dr'
\end{equation*}
Where $L^*$ is the radiance per unit length:
\begin{equation*}
L_*(\mathbf{x}, \vec{\omega}) = \lim_{r \rightarrow 0} \frac{L_r^*(\mathbf{x}, \vec{\omega})}{r}
\end{equation*}
By putting it all together, we obtain this form of the radiative transfer equation:
\begin{equation*}
L_r(\mathbf{x}, \vec{\omega}) = L_r^0(\mathbf{x}, \vec{\omega}) + L_r^*(\mathbf{x}, \vec{\omega})
\end{equation*}
\begin{equation*}
L_r(\mathbf{x}, \vec{\omega}) =  L^0(\mathbf{x}_i, \vec{\omega}) T_r(\mathbf{x}_i, \vec{\omega}) + \int_0^r L_*(\mathbf{x}', \vec{\omega}) T_{r-r'}(\mathbf{x}', \vec{\omega})  dr'
\end{equation*}
We only need now to define $L_*$ in terms of $L$. For this derivation, let us still consider the configuration of Figure \ref{fig:sphere}, where we have a point $\mathbf{x}'$ and vector $\vec{\omega}'$ on the surface of a sphere $d$ surrounding a point $\mathbf{x}$ on path $P_r(\mathbf{x}_i,r)$. We can then define an approximation of $S$ as
\begin{equation*}
S(\mathbf{x}', \vec{\omega}', \mathbf{x}, \vec{\omega}) = \sigma_s(\mathbf{x}', \vec{\omega}', \vec{\omega}) \frac{s}{A_i'} + o(s).
\end{equation*}
$\mathbf{x}', \mathbf{x}$ are two points on the sphere, $s$ is the sphere radius, and $A_i'$ is the area of the sphere that would be lit by a light shined from direction $-\vec{\omega}'$. The term $\sigma_s(\mathbf{x}', \vec{\omega}', \vec{\omega}) = \sigma_s(\mathbf{x}', \vec{\omega}') p(\mathbf{x}', \vec{\omega}', \vec{\omega})$ is the non-normalized phase function. $o(s)$ is an error such as $\lim_{s\rightarrow 0} \frac{o(s)}{s} = 0$.

\begin{figure}
\centering
   \def\svgwidth{0.8\textwidth}
   \input{figures/sphere_geometry.pdf_tex} \\
\caption{Sphere configuration to prove the radiance solution.} %The red rectangle shows where we estimated RMSE in Table \ref{table:quant}.}
\label{fig:sphere}
\end{figure}

We can now recalculate $L_s^*$ \emph{for the sphere configuration} using the approximation:
\begin{equation}
\begin{split}
L_s^*(\mathbf{x}, \vec{\omega}) &= \int_{A_i'} \int_{\Omega^-_i} L(\mathbf{x}', \vec{\omega}')  [\sigma_s(\mathbf{x}', \vec{\omega}', \vec{\omega}) \frac{s}{A_i'} + o(s)] (\vec{n}' \cdot \vec{\omega}') d\omega' dA_i'  \\
&= \int_{4\pi}  L(\mathbf{x}', \vec{\omega}')  [\sigma_s(\mathbf{x}', \vec{\omega}', \vec{\omega}) s + A_i' o(s)] (\vec{n}' \cdot \vec{\omega}') d\omega'\\
&= s \int_{4\pi}  L(\mathbf{x}', \vec{\omega}')  \sigma_s(\mathbf{x}', \vec{\omega}', \vec{\omega}) d\omega' + o(s) \int_{4\pi}  L(\mathbf{x}', \vec{\omega}') A_i'   d\omega'
\end{split}
\end{equation}
Taking the limit for $s\rightarrow 0$, the second term disappears with the $o(s)$, so that we obtain the desired quantity:
\begin{equation}
\label{eq:lstar}
L_*(\mathbf{x}, \vec{\omega}) = \lim_{s\rightarrow 0} \frac{L_s^*(\mathbf{x}, \vec{\omega})}{s} = \sigma_s(\mathbf{x}, \vec{\omega}) \int_{4\pi} L(\mathbf{x}, \vec{\omega}') p(\mathbf{x}, \vec{\omega}', \vec{\omega})   d\omega' 
\end{equation}
Putting it all together, we obtain the \emph{integral form} of the radiative transfer equation:
\begin{equation*}
L_r(\mathbf{x}, \vec{\omega}) =  L^0(\mathbf{x}_i, \vec{\omega}) T_r(\mathbf{x}_i, \vec{\omega}) + \int_0^r \sigma_s(\mathbf{x}', \vec{\omega}) \int_{4\pi} L(\mathbf{x}', \vec{\omega}') p(\mathbf{x}', \vec{\omega}', \vec{\omega})  d\omega' T_{r-r'}(\mathbf{x}', \vec{\omega})  dr'
\end{equation*}
To finish our calculation, we derive the integro-differential form~\ref{eq:rte} from the equation above. We derive the above across $r$, which is the same as a directional derivative $ \vec{\omega} \cdot \nabla$. Of the two terms on the right hand side of the equation above, the first becomes:
\begin{equation}
\begin{split}
\frac{d}{dr} L^0(\mathbf{x}, \vec{\omega}) T_r(\mathbf{x}, \vec{\omega}) &= L^0(\mathbf{x}, \vec{\omega}) \frac{d}{dr}  T_r(\mathbf{x}, \vec{\omega}) \\
&= L^0(\mathbf{x}, \vec{\omega}) (-\sigma_t(\mathbf{x}, \vec{\omega}) T_r(\mathbf{x}, \vec{\omega})) = -\sigma_t(\mathbf{x}, \vec{\omega}) L_r^0(\mathbf{x}, \vec{\omega})
\end{split}
\end{equation}
As for the second term:
\begin{equation}
\begin{split}
\frac{d}{dr} L^*_r(\mathbf{x}, \vec{\omega}) &= \frac{d}{dr} \int_0^r L_*(\mathbf{x}', \vec{\omega}) T_{r-r'}(\mathbf{x}', \vec{\omega})  dr' \\
&= \int_0^r L_*(\mathbf{x}', \vec{\omega}) \frac{d}{dr} T_{r-r'}(\mathbf{x}', \vec{\omega})  dr' + L_*(\mathbf{x}', \vec{\omega})
 \\
 &= -\sigma_t(\mathbf{x}, \vec{\omega})  \int_0^r L_*(\mathbf{x}', \vec{\omega}) T_{r-r'}(\mathbf{x}', \vec{\omega})  dr' + L_*(\mathbf{x}', \vec{\omega}) \\
 &= -\sigma_t(\mathbf{x}, \vec{\omega}) L^*_r(\mathbf{x}, \vec{\omega}) + L_*(\mathbf{x}', \vec{\omega})
\end{split}
\end{equation}
By putting it all together:
\begin{equation}
\begin{split}
\frac{d}{dr} L_r(\mathbf{x}, \vec{\omega}) &= -\sigma_t(\mathbf{x}, \vec{\omega}) [L^0_r(\mathbf{x}, \vec{\omega}) + L^*_r(\mathbf{x}, \vec{\omega})] + L_*(\mathbf{x}', \vec{\omega}) \\
&=  -\sigma_t(\mathbf{x}, \vec{\omega}) L_r(\mathbf{x}, \vec{\omega}) +  L_*(\mathbf{x}', \vec{\omega}) 
\end{split}
\end{equation}
By dropping the dependency on $r$, introducing the definition for $L_*$ (Equation~\ref{eq:lstar}) and the directional derivative symbol, we obtain the integro-differential form of the radiative  transfer equation
\begin{equation*}\vec{\omega} \cdot \nabla L(\mathbf{x}, \vec{\omega}) = - \sigma_t(\mathbf{x}, \vec{\omega}) L(\mathbf{x}, \vec{\omega}) + \sigma_s(\mathbf{x}, \vec{\omega}) \int_{4\pi} L(\mathbf{x}, \vec{\omega}')  p(\mathbf{x}, \vec{\omega}', \vec{\omega}) d \vec{\omega}'\end{equation*}
We can then add an additional term to account for emission, obtaining the form in Equation~\ref{eq:rte}.
\subsection{Global solution to the BSSRDF}
Equation~\ref{eq:rte} only gives a recursive local solution to the radiative transfer problem. We will now derive a new solution that considers the global effects of the BSSRDF for a surface.  To achieve this, we define the operator $\operator{S}^1$:
\begin{equation*}
\operator{S}^1 = \int_0^r \sigma_s(\mathbf{x}', \vec{\omega}) \int_{4\pi} [\ \ ] p(\mathbf{x}', \vec{\omega}', \vec{\omega})  (\vec{n} \cdot \vec{\omega}')  d\omega' T_{r-r'}(\mathbf{x}', \vec{\omega})  dr'
\end{equation*}
Now, we can define a starting radiance $L_0(\mathbf{x}_o, \vec{\omega})$ at a boundary point $\mathbf{x}_o$. We can extend this initial radiance to any point $\mathbf{x} = \mathbf{x}_o + r \vec{\omega}$ of the medium :
\begin{equation*}
L^0(\mathbf{x}, \vec{\omega}) = L^0(\mathbf{x}_o, \vec{\omega}) T_r(\mathbf{x}_o, \vec{\omega})
\end{equation*}
Since we defined $L^0$, we can define a n-ary radiance function $L^{n+1}$ recursively:
\begin{equation*}
L^{n+1} = L^n \operator{S}^1
\end{equation*}
Then, the n-ary radiance can be easily be calculated to be the continuous application of the $S$ operator. If we define $\operator{S}^{n+1} = \operator{S}^1 \operator{S}^n$, we get that 
\begin{equation*}
L^n = L^0 \operator{S}^n
\end{equation*}
For every scattering order $n$.
We can bring this process to infinity by defining the two quantities:
\begin{equation*}
L = \sum_{j=0}^\infty L^j\ \ \ \ \operator{S} = \sum_{j=0}^\infty \operator{S}^j
\end{equation*}
In this particular case, we define $\operator{S}^0 = \operator{I}$, where $\operator{I}$, the identity operator, is a functional for which $f \operator{I} = f$ for every choice of $f$.
That leads to the formulation:
\begin{equation*}
L =  \sum_{j=0}^\infty L^j = \sum_{j=0}^\infty L^0 \operator{S}^j = L^0 \operator{S}
\end{equation*}
It is simple to prove that this formulation satisfies the integral form of the radiative transfer equation:
\begin{equation*}
\begin{split}
L &= L^0 S \\
&= L^0 \left(\operator{I} + \operator{S}^1 + \sum_{j=2}^\infty \operator{S}^j \right) \\
&= L^0 \left( \operator{I} + \operator{S}^1 + \sum_{l=1}^\infty \operator{S}^{l+1} \right) \\
&= L^0 \left( \operator{I} + \left( \operator{I} + \sum_{l=1}^\infty \operator{S}^l \right) \operator{S}^1 \right)  \\
&= L^0 \left( \operator{I} + \operator{S} \operator{S}^1 \right)  \\
&= L^0 + (L^0 \operator{S}) \operator{S}^1   \\
L &= L^0 + L \operator{S}^1   \\
\end{split}
\end{equation*}
Where the last equation corresponds to the integral form of the radiative transfer equation in TODO.

\subsection{Rendering scattering media}
Now, we want to use our formulas to derive a way to render scattering materials. The usual technique used in rendering to solve the rendering equation is Monte Carlo integration with importance sampling. Let us assume we need to solve the integral
$$
I = \int_X f(x) dx
$$
Over some domain $X$. If we sample uniformly samples in $X$, we can define a estimator for $I$:
$$
\hat{I} = \frac{1}{N} \sum_{i=1}^N f(x_i)
$$
From the law of large numbers, $\hat{I} \rightarrow I$ for $N \rightarrow \infty$. To improve the convergence rate, we use importance sampling. If we know a distribution $\text{pdf}(x)$ that is somewhat close in shape to $f(x)$ and from which it is easy to draw samples from. We can rewrite the integral $I$ as :
$$
I = \int_X \frac{f(x)}{\text{pdf}(x)} \text{pdf}(x) dx
$$
Which we can evaluate again using Monte Carlo as:
$$
\hat{I} = \frac{1}{N} \sum_{i=1}^N \frac{f(x_i)}{\text{pdf}(x_i)}
$$
Where the $x_i$ follow the distribution $\text{pdf}(x)$. As before, $\hat{I} \rightarrow I$ for $N \rightarrow \infty$.

Now let us apply importance sampling to obtain a first technique to render scattering materials,  \emph{volume path tracing}. We start by using the integral form of the radiative transfer equation TODO. For the sake of this example, we assume a non emissive homogenous medium, so that the coefficients to not depend on position and direction within the medium. We also assume index matched media, so we can simplify the interaction at the boundary. 

The algorithm, traces a ray $-\vec{\omega}$ from the camera, hitting the surface at some point $\mathbf{x}_o$. We need to evaluate $L_r(\mathbf{x}_o, \vec{\omega})$. We first evaluate $r$ as the distance to the other side of the medium, hitting a point $\mathbf{x}_t = \mathbf{x}_o - r \vec{\omega}$. The transmittance is easily evaluated:
$$
T_r(\mathbf{x}_t, \vec{\omega}) = \exp\left(-\int_0^r \sigma_t dr'\right) = e^{-\sigma_t r}
$$
While $L^0$ can be evaluate by recursively evaluating path tracing in direction $-\vec{\omega}$. Now we want to evaluate the second part of equation TODO. For this, we use Monte Carlo integration with importance sampling. We first reparameterize the integral by imposing $s = r - r'$:
$$
L_r^*(\mathbf{x}_o, \vec{\omega}) = \int_0^r \sigma_s \int_{4\pi} L(\mathbf{x}', \vec{\omega}') p(\mathbf{x}', \vec{\omega}', -\vec{\omega})  d\omega' T_{s}(\mathbf{x}', -\vec{\omega})  ds
$$
Note that $\mathbf{x}' = \mathbf{x}_o - s \vec{\omega} = \mathbf{x}_t + r' \vec{\omega}$. Now we can write the Monte Carlo formulation sampling both integrals at the same time:
$$
L_r^*(\mathbf{x}_o, \vec{\omega}) = \sum_{p=1}^N \frac{\sigma_s L(\mathbf{x}', \vec{\omega}') p(\mathbf{x}', \vec{\omega}', -\vec{\omega}) \exp(-\sigma_t s)}{\text{pdf}(s) \text{pdf}(\vec{\omega}')}
$$
For now, we assume that $s < r$, i.e. we are always sampling a point within the medium. Now, we need to choose the pdfs. For standard phase functions such as Heyney-Greenstein, it is possible to analytically find a way to importance sample a vector $\vec{\omega'}$. So, if the vector is carefully chosen, we have $\text{pdf}(\vec{\omega}') = p(\mathbf{x}', \vec{\omega}', -\vec{\omega})$. Then, we can sample $s$ according to the formula:
$$
s = \frac{-\ln(1 - \xi)}{\sigma_t}
$$
For $\xi \in [0,1)$, that corresponds to a $\text{pdf}(s) = \sigma_t \exp(-s \sigma_t)$. Plugging it in TODO:
$$
L_r^*(\mathbf{x}_o, \vec{\omega}) = \sum_{p=1}^N \alpha L(\mathbf{x}', \vec{\omega}')
$$
Where $\alpha = \sigma_s / \sigma_t = \sigma_s / (\sigma_s + \sigma_a)  $ is called \emph{single scattering albedo}. Note that $0 < \alpha < 1$.  Finally, we can introduce an additional importance sampling. We introduce an absorption probability, so that the contribution from $L(\mathbf{x}', \vec{\omega}')$ is included with probability $\text{pdf}_a$. If we choose $\text{pdf}_a = \alpha$, we get to our final formulation:
$$
L_r^*(\mathbf{x}_o, \vec{\omega}) = \sum_{p=1}^N \alpha \frac{L(\mathbf{x}', \vec{\omega}')} {\text{pdf}_a} = \sum_{p=1}^N  L(\mathbf{x}', \vec{\omega}') 
$$
Which means, we just need to recursively evaluate radiance to get the final result. At the next recursive step, we will evaluate a new $s$ and $\vec{\omega}'$, interrupt the process with probability $\alpha$ and so on. This creates a random walk within the medium, that does not terminate for $s < r$. If $s > r$, it means that we are exiting the medium. In this case, we continue path tracing from the exit point in the last evaluated direction $\vec{\omega}'$.

\subsection{BSSRDF Models}
As we did in the previous section, we distinguish between two cases: analytic and discrete models. In the case of analytical models, Nicodemus was the first to give a modern definition of BSSRDF, with Jensen et al. to give the first BSSRDF model, also called the standard dipole. We will outline the main points in deriving an analytical BSSRDF here, referencing to our note TODO for more details on derivation. We will derive three models based on the diffusion approximation: the standard dipole, the better dipole and the directional dipole.

Generically, the light exiting at a point $\mathbf{x}_o$ on a direction $\vec{\omega}_o$ on a scattering medium comes from three contributions: direct transmission, single and multiple scattering. Direct transmission is the contribution of the light coming from direction $-\vec{\omega}_{21}$ that does not scatter along the path. The single scattering term comes from light that enters at any point of the medium, scatters once, then aligns with $\vec{\omega}_{21}$ to refract out in direction $\vec{\omega}_o$. The multiple scattering term is similar, but it includes light that has scattered more than once before exiting the medium. 

The standard and better dipole model multiple scattering only, while the directional dipole includes also part of the single scattering contribution. In the first two cases, single scattering needs to be handled separately to achieve a proper result. 

At the core, these dipoles model the scattering of light as a diffusion process. Diffusion is a way to approximate a highly complicated stochastic process (such as a light transport simulation) with a structured mathematical description. 

As before, we define a couple of functionals to simplify notation:
\begin{equation*}
G = \int_{4\pi} [\ \ ] d\omega; \ \ \ \vec{G} = \int_{4\pi} [\ \ ] \vec{\omega} d\omega
\end{equation*}
In radiative transport, we use spherical harmonics to simplify radiance:
\begin{equation*}
L(\mathbf{x}, \vec{\omega}) = \frac{1}{4\pi}\phi(\mathbf{x}) + \frac{3}{4\pi} \vec{E}(\mathbf{x}) 
\end{equation*}
Where $\phi = L G$ and $\vec{E} = L \vec{G}$. Once we combine the radiative transfer equation and the diffusion approximation, we get the form:
\begin{equation*}
(D \nabla^2 - \sigma_a)\phi(\mathbf{x}) = -q(\mathbf{x}) + 3D \nabla \cdot \vec{Q}(\mathbf{x}, \vec{\omega})
\end{equation*}
Where $q = \epsilon G$ and $\vec{Q} = \epsilon \vec{G}$ are the integrals of the emission term in the RTE $\epsilon(\mathbf{x}, \vec{\omega})$. $q$ is also called the \emph{source term}. Depending on the choice of source term, different approximations can be obtained. Standard and better dipole use a point source term, while the directional dipole uses a ray source. Equation TODO is a particular case of screened Poisson equation, that can be explicitly integrated. For the standard dipole, we obtain:
\begin{equation*}
\phi(\mathbf{x}) = \frac{\alpha' \Phi_i}{4 \pi D} \frac{e^{-\sigma_{tr} r}}{r}
\end{equation*}
And for the directional solution, we get:
\begin{equation*}
\phi(\mathbf{x}) = \frac{\alpha' \Phi_i}{4 \pi D} \frac{e^{-\sigma_{tr} r}}{r} \left(1 + 3D \frac{1 + \sigma_{tr} r}{r} \cos\theta \right)
\end{equation*}
Now that we have a formulation for the fluence, we need to connect it to the BSSRDF. First, we consider only the diffusive part of the BSSRDF $S_d$, i.e. the part depending on the diffusion approximation. From the definition of BSSRDF TODO we get to this formulation:
\begin{equation*}
S_d(\mathbf{x}_i, \vec{\omega}_i, \mathbf{x}_o)  =  \frac{1}{4\pi T_{12}C_\phi(1/\eta)} \frac{d M_d(\mathbf{x}_o)}{d \Phi_i(\mathbf{x}_i, \vec{\omega}_i)} 
\end{equation*} 
Where the diffusive exitant radiance $M_d$ is approximated thourgh the diffusion approximation as:
\begin{equation*}
M_d(\mathbf{x}_o) =  C_\phi(\eta) \phi(\mathbf{x}_o) - C_E(\eta) D \nabla\phi(\mathbf{x}_o) \cdot \vec{n}_o
\end{equation*}
Where $T_{12}$ is the incoming Fresnel coefficient, $C_\phi$ amnd $C_E$ are the first two order fresnel integrals, that can be approximated with analytical formulas. By plugging in our solutions for the fluence, we get the final formulation for the BSSRDFs. In the case of Jensen dipole, we further simplify the Fresnel integrals so that $C_\phi(\eta) = 0$, $C_E(\eta) = 1$ and $C_\phi(1/\eta) = 1/4$. If we define $\mathbf{x} = \mathbf{x}_o - \mathbf{x}_i$, the Jensen bssrdf becomes:
\begin{equation*}
S_d(\mathbf{x}_i, \mathbf{x}_o)  =  \frac{\alpha'}{4 \pi^2} \frac{e^{-\sigma_{tr} r}}{r^3} (1 + \sigma_{tr} r) \mathbf{x} \cdot \vec{n}_o 
\end{equation*}
The better dipole becomes:
\begin{equation*}
S_d(\mathbf{x}_i, \mathbf{x}_o)  = \frac{1}{4\pi C_\phi(1/\eta)} \frac{\alpha'}{4 \pi^2} \frac{e^{-\sigma_{tr} r}}{r^3} \left[ C_\phi(\eta) \frac{r^2}{D} + C_E(\eta) (1 + \sigma_{tr} r) \mathbf{x} \cdot \vec{n}_o \right]
\end{equation*}
And the directional dipole:
\begin{multline*}
S_d(\mathbf{x}_i, \vec{\omega}_i, \mathbf{x}_o)  = \frac{1}{4\pi C_\phi(1/\eta)} \frac{1}{4 \pi^2} \frac{e^{-\sigma_{tr} r}}{r^3} \bigg[ C_\phi(\eta) (\frac{r^2}{D} +  3 (1 + \sigma_{tr} r) \mathbf{x}\cdot\vec{\omega}_{12} ) \\ - C_E(\eta) \left[3D (1 + \sigma_{tr} r) \vec{\omega}_{12} \cdot \vec{n}_o - \left((1 + \sigma_{tr} r) + 3D \frac{3 (1 + \sigma_{tr} r)  + (\sigma_{tr} r)^2}{r^2}\mathbf{x}\cdot\vec{\omega}_{12}\right) \mathbf{x} \cdot \vec{n}_o\right] \bigg]
\end{multline*}
All these solutions are suitable only for an infinite medium with coefficients $\sigma_s$, $\sigma_a$ and $g$. To get a full BSSRDF, we need to introduce a boundary. For the vast majority of analytical BSSRDFs, we derive a solution for a semi infinite plane configuration, then correcting the formulas for non planar geometry. 

To achieve a solution for a point $\mathbf{x}_o$ on a semi infinite plane with normal $\vec{n}_o$, we set the net inward flux on point $\mathbf{x}_o$ zero. This leads for the following boundary condition for the diffusion approximation: 
\begin{equation*}
\phi(\mathbf{x}_o) - 2 A D (\vec{n}_o \cdot \nabla) \phi(\mathbf{x}_o) = 0
\end{equation*}
Where $A$ is a term that accounts for mismatched index of refraction boundaries. This condition can be satisfied by introducing two light sources, called the \emph{real} and \emph{virtual} sources, that together form a \emph{dipole configuration}. Standard and better dipole use the configuration in Figure TODO, with a real point source above and one virtual point source below the surface. The directional and forward dipoles use a real ray source on the surface, and a virtual ray source above the surface. The infinite solution dipole is then evaluated for real and virtual source, then the results are subtracted to satisfy the boundary condition and obtain the final BSSRDF value for rendering. 


\subsection{BRDF Models}

Some simplifications can then be introduced to obtain more tractable functions. We assume that the BSSRDF is limited across a small area around the emergence point $\mathbf{x}_o$, and zero everywhere else. This is the case for a particular set of materials, such as plastic or metals. In this configuration, we can assume that the radiance is constant across the plane ($L_i(\mathbf{x}_i, \vec{\omega}_i) \approx L_i(\vec{\omega}_i)$). We also need to assume the material to be locally isotropic, i.e. its properties do not change across the surface. In this case, we can approximate the outgoing radiance as:
\begin{equation*}
d L_r(\mathbf{x}_o, \vec{\omega}_o) \approx \int_A d L_o(\mathbf{x}_i, \vec{\omega}_i, \mathbf{x}_o, \vec{\omega}_o) = \int_A S(\mathbf{x}_i, \vec{\omega}_i, \mathbf{x}_o, \vec{\omega}_o) d \Phi_i(\mathbf{x}_i, \vec{\omega}_i) 
\end{equation*}
By the definition of flux, irradiance and the assumption of constant radiance:
\begin{equation*}
\begin{split}
d L_r(\mathbf{x}_o, \vec{\omega}_o) &\approx \int_A S(\mathbf{x}_i, \vec{\omega}_i, \mathbf{x}_o, \vec{\omega}_o) L_i(\mathbf{x}_i, \vec{\omega}_i) (\vec{n} \cdot \vec{\omega}_i) d A_i d \omega_i  \\ &= L_i(\vec{\omega}_i) (\vec{n} \cdot \vec{\omega}_i) d \omega_i \int_A S(\mathbf{x}_i, \vec{\omega}_i, \mathbf{x}_o, \vec{\omega}_o)   d A_i \\ &= d E_i(\vec{\omega}_i) \int_A S(\mathbf{x}_i, \vec{\omega}_i, \mathbf{x}_o, \vec{\omega}_o) d A_i
\end{split}
\end{equation*}
The last integral is a function purely dependent on the two angular vectors $\vec{\omega}_i$ and $\vec{\omega}_o$, and it becomes the proportionality constant between incoming irradiance and outgoing radiance:
\begin{equation*}
d L_r(\mathbf{x}_o, \vec{\omega}_o) = f_r(\mathbf{x}_o, \vec{\omega}_i, \vec{\omega}_o) d E_i(\mathbf{x}_o, \vec{\omega}_i)
\end{equation*}
 The new function $f_r$ is called \emph{Bidirectional reflectance distribution function} (BRDF). The BRDF is measured in $\siunitnospace{\per \steradian}$. As before, we can obtain the outgoing radiance from the definition above:
\begin{equation*}
L_r(\mathbf{x}, \vec{\omega}) = \int_\Omega f_r(\mathbf{x}, \vec{\omega}_i,  \vec{\omega}_o) L_i(\mathbf{x}, \vec{\omega}_i) (\vec{n} \cdot \vec{\omega}_i) d\vec{\omega}_i  \siunit{\watt \per \square \metre \per \steradian}
\end{equation*}

\subsection{Empirical BRDFs}
TODO

\section{Fast rendering techniques} 
\subsection{Rasterization}
The first of the two techniques is rasterization. In this technique, the rendering primitives, usually triangles, are scanned one by one and then drawn in the corresponding occupied area of the screen. Scanline rendering can then be used to transform each triangle into multiple pixel-size elements, also called \emph{fragments}. 

By its nature, rasterization can be highly parallelized, since every primitive can be drawn in parallel. On modern graphics cards, the rasterization process is part of the graphics pipeline, a highly fine tuned process to render triangles efficiently. A full version of the pipeline is illustrated in Figure TODO. The pipeline is composed of programmable parts (called \emph{shaders}) and hardware parts, that are only controllable through state flags. The core rasterization process, transforming primitives into pixels, is executed in parallel by a specified hardware unit. 

Let us describe the life of a single triangle through the pipeline. We first take a triangle, that is composed by three vertices. For each vertex, we execute a programmable part, called the \emph{vertex shader}, that allows operation such as model and perspective transformation. Once the vertices have been processed, the primitive is assembled and then processed by the hardware rasterizer, generating a certain number of fragments. For each fragment, we execute another programmable shader, the \emph{fragment shader}. This stage outputs the final color of the fragment, so in this stage we usually execute the proper shading of the fragment, including light interaction. Multiple attributes can be passed inbetween vertex and fragment shader, that will be interpolated using the triangle's barycentric coordinates. This allows to interpolate attributes, such as normal and texture coordinates, across the triangle, leading to a more accurate appearance. 

Once the fragments are generated, they need to be stored on the image plane. Note that multiple fragments can land on the same pixel, and that the order of landing of the fragments can be different, since the full pipeline is entirely asynchronous. So, modern GPUs use another hardware unit, the Render OutPut unit (ROP) to decide how to store the fragments in the final image. This unit usually can perform depth testing or blending of the fragments, allowing us to obtain a consistent result across frame invocations.

The full graphics pipeline is more complicated, allowing tessellation (domain and hull shaders), geometry manipulation (geometry manipulation) or writing back into a vertex stream (transform feedback). More recently, we can use shaders that are completely detached from the graphics pipeline (compute shaders), that allows us to perform tasks in parallel that are not directly related to the graphics pipeline. 

Rasterization is extensively used in game development pipelines, given its high speed and predictability. However, it is not particularly well suited to propagate light across a scene, making it difficult to achieve complex optical effects, that often require solutions on a case by case basis.

\subsection{Ray tracing}
Ray tracing can be considered the dual technique to rasterization. In rasterization, we match each triangle to its final position on the screen. In ray tracing, we do the opposite: for each pixel on the screen, we find the corresponding triangles that land within that pixel. This equates to \emph{shooting} a ray through the pixel, and intersecting it though the scene geometry. Once the hitpoint is found, attributes can be generated and the point shaded as in a fragment shader. Generically, the rays can be further traced down the scene, allowing this technique to simulate complex optical phenomena, such as light propagation. 
Modern GPU ray tracers build a tree data structure to efficiently lookup intersections. The most commonly used data structure is a bounding volume hierarchy, or BVH. In a BVH, each of the primitives (usually triangles, but custom primitives such as spheres, cubes or quadrilaterals are possible) is enclosed in a bounding box. The various bounding boxes are then arranged in a tree data structure for fast lookup. Once a ray is traced, it is first inexpensively tested against the bounding boxes. For the bounding boxes that are hit, a expensive intersection test is performed to check for a hit. Depending on the shading algorithm, a callback can be issued at every intersection (\emph{any hit}), or at the intersection closest to the camera (\emph{closest hit}). 

Many challenges need to be solved by an efficient GPU ray tracer. The first challenge is recursion. In the case of an algorithm such as path tracing, a recursive ray tracing operation is performed at the closest hit. So, an efficient ray traced needs to keep track of the state at each intersection, generically though a recursion stack. The second challenge is on how to manage the data structure if the geometry of the material changes, through deformation or animation. Simple rigid tranformations can be handled relatively inexpensiley by the BVH, tough more complicated operation need to be done if the change is at the primitive level. Modern ray tracing techniques such as the TrBVH can rebuild part of the BVH on the fly without excessive memory consumption. 

Given its state as a "natural" solution to light transport and its ability to model complex optical effects, ray tracing is commonly employed in the movie industry. Moreover, ray tracing scales better to scenes with a huge amount of triangle, such as animated movies or VFX scenes. On the other hand, ray tracing is limited in the game developers community, due to smaller scenes and increased memory and lookup footprint. Moreover, algorithms such as path tracing tend to give a noisier result, that is not acceptable in game applications. 


